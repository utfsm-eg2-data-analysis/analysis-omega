\chapter{Event Selection}

All particles are organized in rows and we gather information for all detectors banks.

\section{Electron identification}

The detected particle is identified as an electron if it satisfies the following requeriments.

It is a challenge.

\textbf{At the trigger level.} It is required that the elctron has a minimum amount of energy in the electromagnetic calorimenter EC in coincidence with a signal in the Cherenkov Counter (CC).

\textbf{Geometrical match.} It is also required to have a $DC \times CC$ smaller than $\approx 5^\circ$.



\textbf{Charge.} Must be -1.

\textbf{Regarding the banks.} EVNT status is between 0 and 100. And the number of rows found in the CCPB, ECPB, SCPB banks is different from 0 and their respective status above 0.

\textbf{Momentum.} Orlando says that it should be $P > 0.75$, because momentum trigger? Which is in agreement with Taya's AN for the DIS cuts (and the $y < 0.85$ cut). Mike agree with $P > 0.64$. I think all three researches agree here.

\textbf{Inner stack energy cut.} All three researches agree with $E_{in} > 0.06$ and $E_{out} > 0$ .
	    
\textbf{EC Fiducial cut.} Orlando proposes a target and sector dependent EC fiducial cut. While Taya and Mike propose a simpler one.
\begin{equation}
    40 < U < 400, \quad 0 < V < 360, \quad 0 < W < 390 \, (\mbox{cm})
\end{equation}

\textbf{EC and SC coincidence time.} Where $c = 30$ (cm/ns) and $\sigma = 0.35$ (ns). Mike proposed $3\sigma$. And Taya uses a different approximation.
\begin{equation}
    \left| t_{EC} - \left( t_{SC} - \frac{l_{EC} - l_{SC}}{c} \right) \right| < 5 \, \sigma (\mbox{ns})   
\end{equation}

\textbf{Number of CC photoelectrons detected.} This coresponds to the number of photelectrons produced by the electron candidate in the Chernekov Counters (CC). Is sector dependent.
\begin{itemize}
\item For sectors 0 and 1, $N_{phe} > 25$.
\item For sector 2, $N_{phe} > 26$.
\item For sector 3, $N_{phe} > 21$.
\item For sectors 4 and 5, $N_{phe} > 28$.
\end{itemize}

\textbf{Sampling Fraction cut.} Objective: prevent misidentification, distinguish between $\pi^-$ and $e^-$.. This is a cut of $2.5 \sigma$ around sampling fraction fit, just as Taya's thesis. I believe Orlando uses the same one. Sector and target dependent.
\begin{equation}
  \left| \frac{E_{tot}}{P} - \mu(P) \right| > 2.5 \, \sigma(P)
\end{equation}

\textbf{Sampling Fraction cut - more methods.} The first method consists in analysing? the distribution of $E_{total}$ versus electron momentum. This 
\begin{equation}
  \frac{E_{tot}}{0.27 \times 1.15} - 0.2 < P < \frac{E_{tot}}{0.27 \times 1.15} + 0.4
\end{equation}
The second method consists in analysing? the dependence/relationship of the ratios $E_{inner}/P$ and $E_{outer}/P$.
\begin{equation}
  0.8 \times 0.27 \times P < E_{in} + E_{out} < 1.2 \times 0.27 \times P
\end{equation}

\textbf{DIS cuts.} A posterior cut is made:
\begin{itemize}
\item $Q^2 > 1$
\item $W > 2$, to exclude the nucleon resonances
\item $y = \frac{\nu}{E} < 0.85$, limit the magnitude of radiative corrections
\end{itemize}
%\begin{figure}
%  \centering
%  
%\end{figure}
%%%%%%%%%%%%%%%%%%%%%%%%%%%%%%%%%%%%%%%%%%%%%%%%%%%%%%%%%%%%%%%%%%%%%%%%%%%%%%%%%%%%%%%%%%%%%%%%%%%%%%
\section{$\pi^{+}$ identification}

All $\pi^{+}$ must satisfy the following requeriments:

\textbf{Charge.} Must be 1.

\textbf{Banks status.} Should be detected in EVNT and DCPB. (Status above 0)

\textbf{High energy $\pi^+$.} With $P \ge 2.7$. A Cherenkov Counter (CC) identification technique is used. (CC status above 0).
\begin{itemize}
\item $N_{phe} > 25$
\item $\chi^2 < 5/57.3$
\end{itemize}
But, how do we get rid of positrons?

\textbf{Low energy $\pi^+$.} With $P < 2.7$. Here is used a time-of-flight (TOF) identification technique. It should exist a valid SC status. We can not determine $\beta$, due to experimental time and momentum resolution.
\begin{equation}
  \beta = \frac{P}{E} = \frac{c}{\sqrt{1 + \frac{m^2}{c^2}}}
\end{equation}


\begin{itemize}
    \item $0.0 < P \le 0.25$ and $-1.45 \le \beta < 1.05$
    \item $0.25 < P \le 0.5$ and $-1.44 \le \beta < 1.05$
    \item $0.5 < P \le 0.75$ and $-1.44 \le \beta < 1.05$
    \item $0.75 < P \le 1.0$ and $-1.4 \le \beta < 1.05$
    \item $1.0 < P \le 1.25$ and $-1.35 \le \beta < 1.03$
    \item $1.25 < P \le 1.5$ and $-1.35 \le \beta < 0.95$
    \item $1.5 < P \le 1.75$ and $-1.35 \le \beta < 0.87$
    \item $1.75 < P \le 2.0$ and $-1.25 \le \beta < 0.68$
    \item $2.0 < P \le 2.25$ and $-0.95 \le \beta < 0.65$
    \item $2.25 < P \le 2.5$, $-1.05 \le \beta < 0.61$ and $m^2 < 0.5$
    \item $2.5 < P \le 2.7$, $-1.05 \le \beta < 0.61$ and $m^2 < 0.4$
\end{itemize}

%%%%%%%%%%%%%%%%%%%%%%%%%%%%%%%%%%%%%%%%%%%%%%%%%%%%%%%%%%%%%%%%%%%%%%%%%%%%%%%%%%%%%%%%%%%%%%%%%%%%%%
\section{$\pi^{-}$ identification}

To identify the particle as a $\pi^{-}$, it must satisfy the following requeriments:

\textbf{Time correction.} Depends on the particle.
\begin{equation}
  \beta = \left(\frac{l_{SC}}{30} - t_{SC}\right)_e - \left(\frac{l_{SC}}{30} \sqrt{\frac{M^2}{P^2} + 1} - t_{SC} \right)_{k} - 0.08
\end{equation}     
    
\begin{itemize}
\item $ charge = -1 $
\item $ 0 < Status(k) < 100$
\item DC status $ > 0 $
\item $ E_{tot} < 0.15 \quad\&\quad E_{in} < (0.085 - 0.5 E_{out}) $
\item $  $
\item $  $
\item $  $
\item $  $
\item $  $
\end{itemize}
    
\textbf{EC fiducial cuts.}
\begin{equation}
    40 < U < 400, \quad V < 360, \quad W < 390 \, (\mbox{cm})
\end{equation} 

\begin{equation}
  E_{tot} < 0.15
  E_{in} < 0.085 - 0.5*E_{out}
  !StatCC > 0
  Nphe > 25
  0 < P \& P \le 0.5 \& T4 > -0.87 \& T4 < 0.63
  \parallel 0.5 < P \& P \le 1.0 \& T4 > -0.55 \& T4 < 0.37
  \parallel 1.0 < P \& P \le 1.5 \& T4 > -0.55 \& T4 < 0.38
  \parallel 1.5 < P \& P \le 2.0 \& T4 > -0.60 \& T4 < 0.44
  \parallel 2.0 < P \& P \le 2.5 \& T4 > -1.00 \& T4 < 0.45
  \parallel 2.5 < P \& P \le 3.0 \& T4 > -1.00 \& T4 < 0.40
  3.0 < P \& T4 > -2.00 \& T4 < 0.45
\end{equation}

%%%%%%%%%%%%%%%%%%%%%%%%%%%%%%%%%%%%%%%%%%%%%%%%%%%%%%%%%%%%%%%%%%%%%%%%%%%%%%%%%%%%%%%%%%%%%%%%%%%%%%

\section{$\gamma$ identification}

Let's identify photons!

\textbf{Charge.} Must be 0.

\textbf{Momentum.} So far, I've got no momentum cut. Mike requires $P > 0.15$. Could be this one?
\begin{equation}
    \frac{\max{(E_{tot}, E_{in} + E_{out})}}{0.273} > 0.1
\end{equation}

\textbf{EC Fiducial cut.} Here, all three researchers agree.
\begin{equation}
    40 < U < 410, \quad 0 < V < 370, \quad 0 < W < 410 \quad (\mbox{cm})
\end{equation}

\textbf{Time cut on $\beta$.} A $\gamma$ should travel at the speed of light. Orlando uses this:
\begin{equation}
    -2.2 < t_{EC} - \frac{l_{EC}}{30} < 1.3
\end{equation}
But here's a subtletie. Taya mentions a $t_{start}$ variable, taken from the EVNT bank. While $t_{EC}$ and $l_{EC}$ come from the ECPB bank. And it should correspond to
\begin{equation}
    0.935 < \beta < 1.95
\end{equation}
Mike requires something simpler.
\begin{equation}
    0.95 < \beta < 1.95
\end{equation}
So, all three researchers agree.

\textbf{Timing.} Exclusive from Mike. This is the coincide time between the EC and what? Maybe that's $t_{start} = 54$ (ns).
\begin{equation}
    \left| \left( t_{EC} - \frac{l_{EC}}{c}\right) - 54 \, \mbox{(ns)} \right| < 3 \times 5.3 (\mbox{ns})
\end{equation}

\textbf{SC $m^2$.} Exclusive from Mike. With $m^2$ calculated from TOF.
\begin{equation}
    -0.041537 < m^2_{\gamma_1} < 0.031435
\end{equation}
\begin{equation}
    -0.044665 < m^2_{\gamma_2} < 0.035156
\end{equation}

\textbf{Energy correction.} Energy (and target) dependent correction!
\begin{equation}
    1.129-0.05793/E-1.0773e-12/(E*E) for C, Pb 
    1.116-0.09213/E+0.01007/(E*E) for Fe
\end{equation}

\textbf{Momentum correction.} Using the ECPB bank treatment, studied by Taya and Orlando.

%%%%%%%%%%%%%%%%%%%%%%%%%%%%%%%%%%%%%%%%%%%%%%%%%%%%%%%%%%%%%%%%%%%%%%%%%%%%%%%%%%%%%%%%%%%%%%%%%%%%%%
\section{$\pi^0$ identification}

A $3\sigma$ cut arount $\pi^0$ invariant mass, after a Gaussian fit over the signal, is done.

% \begin{figure}
%  \centering
%  
%\end{figure}

%%%%%%%%%%%%%%%%%%%%%%%%%%%%%%%%%%%%%%%%%%%%%%%%%%%%%%%%%%%%%%%%%%%%%%%%%%%%%%%%%%%%%%%%%%%%%%%%%%%%%%

\section{$\omega$ identification}

%%%%%%%%%%%%%%%%%%%%%%%%%%%%%%%%%%%%%%%%%%%%%%%%%%%%%%%%%%%%%%%%%%%%%%%%%%%%%%%%%%%%%%%%%%%%%%%%%%%%%%

\section{Binning}

%%%%%%%%%%%%%%%%%%%%%%%%%%%%%%%%%%%%%%%%%%%%%%%%%%%%%%%%%%%%%%%%%%%%%%%%%%%%%%%%%%%%%%%%%%%%%%%%%%%%%%

%%% Local Variables:
%%% mode: latex
%%% TeX-master: "../main"
%%% End:
