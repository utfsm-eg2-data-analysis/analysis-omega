\chapter{Monte Carlo Simulations}

To get reliable physisc analysis results, it s ncessary to calculate the acceptance of the expeirmental setup and apply the obtained correction factor to teh data.

\section{Generation and reconstruction of MC events}

The process:
\begin{itemize}
\item LEPTO \cite{lepto}
\item GSIM \cite{gsim}, it creates an idealized software model of the CLAS. It is built in base of geant \cite{geant}. It allows to model the response of the spcetrometer to the passage of particles thourgh it, including process such as: energy loss and radiation of particles during transport through different parts of CLAS. \textbf{Input:} set of four-momentum of particles created by the event generator.
\item \textbf{GPP:} gsim post processor program is used to remove signals from dead wirtes in the DC and bad tubes in the SC.
\item recsis or \verb|user_ana|: this program is in cahreg of reconstructing the generated events. It was built using the same librarries that were used forprocessing the actual data from the EG2 running period. 
\end{itemize}
In the final stage, the same cuts ued on reconstructed were also applide to the data to determine the acceptance.

\subsection{LEPTO}
\label{sec:lepto}

Here it is a list of the parameters set in LEPTO.
\section{Acceptance}

Idea for plot: generated, data and reconstructed for theta vs phi distribution, showing sectors and edges. Very educational!

\textbf{Reasons:}
\begin{itemize}
\item polar and azimuthal angular coverage is not perfect/full?
\item material of the CLAS?
\item different size, shape and position of targets?
\item target system support structures
\end{itemize}
all of these may cause secondary scatterrings.

Definition of Acceptance: is done in a bin-by-bin basis. 
\begin{equation}
  A \equiv \frac{N_{rec}(Q^2, \nu, Z, p_T^2)}{N_{gen}(Q^2, \nu, Z, p_T^2)}
\end{equation}

Bin migration effects: as the resolution of the detector is finite. It is possible that an generated event may be reonstructed in another bin.

%%% Local Variables:
%%% mode: latex
%%% TeX-master: "../main"
%%% End:
