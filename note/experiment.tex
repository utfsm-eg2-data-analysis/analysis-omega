\chapter{Experimental setup}
\label{ch:experiment}
The experiments (E02-104, E02-110) were carried out during the run period called EG2, from January 9 2004 to March 5 of 2004 in the experimental Hall-B. The run was divided in three, labeled {\it a,b,c}. The analysis presented here is over the data of EG2c with electron beam energy 5.014 GeV. The standard CLAS instrumentation is used and a unique double-target design was made in order to minimize the systematic uncertainties by the simultaneous exposure of the two targets. The first target is liquid deuterium and the second target is a solid material (C,Al,Fe,Sn,Pb).The liquid and solid targets were separated by the aluminium foil used for beam tuning. The separation distance between the two targets is 4 cm. The length of the liquid  target was 2 cm, while the solid targets had the form of a disk each with a radius of 0.15 cm but with different thicknesses: 0.014 cm, 0.04 cm and 0.17 cm for Pb, Fe and C respectively. The solid and the liquid targets were placed backwards from the CLAS sector, at 25 cm and 30 cm respectively. The target position was chosen to increasing acceptance for the negatively charged particles as the orientation of CLAS torus polarity in this experiment was such that negative particles were inbending. The double target system with all the support structures was fully implemented in GSIM \cite{target}.

The solid targets used in the present analysis are $^{12} C$, $^{56} Fe$ $^{208} Pb$. The statistic for the different target configurations is shown in table \ref{tb:target_stat}
\begin{table}
\begin{center}
\begin{tabular}{|l|l|l|l|}
\hline
Targets&D + C&D + Fe&D + Pb\\
\hline
Luminosity (10 34 1/cm 2 · s)& 1.3 & 2.0 & 1.3\\
\hline
Number of triggers (10 9 ) & 1.0 & 2.0 & 1.5\\
\hline
\end{tabular}
\caption{Statistic for the different target configurations of the EG2c run that were used.}
\label{tb:target_stat}
\end{center}
\end{table}

%