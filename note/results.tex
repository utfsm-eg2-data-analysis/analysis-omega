\chapter{Results}
\label{ch:results}
Taking into consideration that both targets (the solid and liquid) are exposed to the beam at the same time, thus having the same luminosity, and using the relation:
\begin{equation}
\sigma\mathcal{L} = \mathcal{R},
\end{equation}
where $\sigma$ is the cross section, $\mathcal{L}$ the luminosity and $\mathcal{R}$ the event rate, the multiplicity ratio can be measured as follows:
\begin{equation}
R_D^A=\frac{\lrp{\frac{N_h(Q^2,\nu,p_T^2,Z,\phi_{pq})}{Ne(Q^2,\nu)}}_A}{\lrp{\frac{N_h(Q^2,\nu,p_T^2,Z,\phi_{pq})}{Ne(Q^2,\nu)}}_D}
\end{equation}
Where $N_h(\cdot)$ and $Ne(\cdot)$ are the number of hadrons and number of electrons in the different kinematical regions given by the binning considered in the previous chapter.

The multiplicity ratio including acceptance correction is:
\begin{equation}
R(b) = \frac{\left(N_h(b)/\epsilon(b)/N_e(Q^2,\nu)\right)_A}{\left(N_h(b)/\epsilon(b)/N_e(Q^2,\nu)\right)_D}
\label{eq:MR_AC}
\end{equation}
The dependence $b$ is used for the sake of simplicity and represents certain kinematical bin (i.e. $b = Q^2,\nu,p_T^2, z $). $\epsilon$ \eqref{eq:acceptance} is the acceptance in that bin .
%
\begin{equation}
\epsilon(b) = N_{acc}(b)/N_{th}(b),
\label{eq:acceptance}
\end{equation} 
where $N_{acc}$  and $N_{th}$ are the number of events accepted and thrown, see Fig. %\ref{fig:MR_pi0_ptq2nu}
%
\subsection{Multiplicity Ratio Eta 2$\gamma$ decay}
The analysis shows the typical characteristics of the hadrons production in a nuclear environment, see for example: \cite{Hermes_2007},\cite{Hermes_2011} and \cite{Hermes_2013}. Though the results are not precise due to the lack of statistics to process correctly the results. Here the background subtraction is the main problem and can be improved by making a multidimensional analysis. Nevertheless the suppression observed is quite strong compared to other hadrons.

The available Iron data is more than the double of the Carbon and nearly twice the statistics for the Lead target, only this target shows a clear pattern. Consider the iron curves produced on $\eta \rightarrow \pi^+\pi^-\pi^0$, Fig. \ref{fig:eta_3pi_Fe}. The background in this decay channel is not very strong due to the limited available phase space, since there is 5 particles required in the final state.

It shows attenuation with $z$ and a Cronin effect with $P_t^2$. Not a clear dependency on $Q^2$ neither $\nu$ is observed.
%
%
\begin{figure}[H]
\centering
\includegraphics[scale=0.25]{eta_3pi_Fe_summary.png}
\caption{Multiplicity ratio for Iron data on $\eta \rightarrow \pi^+\pi^-\pi^0$.}
\label{fig:eta_3pi_Fe}
\end{figure}
%
Taking this analysis further, we can make a similar estimation of the behaviour of the eta particles by making a cut on the invariant mass spectrum of the $\eta \rightarrow 2\gamma$, by inspection of the data a reasonable cut will be $0.45<m(\gamma\gamma)<0.7$, a cut on  $0.1<m(\gamma\gamma)<0.18$ is performed to contrast the $\pi^0$ behaviour. This will be a part of a rough particle identification.

Considering that the produced background for solid and liquid target should be similar, the yield ratio is mainly influenced by the physics phenomenon, hence the rough particle identification shows the main characteristics.

As expected, the results for $\pi^0$ are very close to the others obtained previously, see Fig. \ref{fig:MR_pi0_z_rough} and \ref{fig:MR_pi0_pt2_rough}.

\begin{figure}[H]
\centering
\includegraphics[scale=0.5]{MR_z_pi0_rough.pdf}
\caption{Multiplicity ratio for $\pi^0$, $z$ dependency. Previous results contrasted with the  rough particle identification $0.1<m(\gamma\gamma)<0.18$.}
\label{fig:MR_pi0_z_rough}
\end{figure}

\begin{figure}[H]
\centering
\includegraphics[scale=0.5]{MR_pt2_pi0_rough.pdf}
\caption{Multiplicity ratio for $\pi^0$, $P_t^2$ dependency. Previous results contrasted with a rough estimate.}
\label{fig:MR_pi0_pt2_rough}
\end{figure}

In the same line, the results for $\eta \rightarrow \gamma\gamma$ are shown on Fig. \ref{fig:MR_eta_aa_rough}. Here, an attenuation depending on the nuclear environment is clearly observed, the acceptance correction on the result will slightly change the behaviour of the curves but not atomic mass hierarchy.

Similar to the neutral pion, the  $\eta$ multiplicity ratio does not exhibit a dependency on $Q^2$, in contrast to the weak dependency on $z$ and $\nu$. There is a lower limit on the minimum $z$ value due to the mass of the $\eta$, $z_{min}\approx 0.55/\nu_{max} \approx 0.125$. This limit together with the energy resolution doesn't allow measurement below $z=0.5$.
%
\begin{figure}[H]
\centering
\includegraphics[scale=0.25]{eta_aa_rough.png}
\caption{Multiplicity ratio for $\eta \rightarrow \gamma\gamma$. Using a rough estimation.}
\label{fig:MR_eta_aa_rough}
\end{figure} 
% 
\section{Comparison between $\pi^0$ and $\eta$ }
The estimation obtained using the cut on the mass spectrum ( $m(\gamma \gamma )$) is used in the comparison with TM analysis of $\pi^0$. The Fig. \ref{fig:MR_eta_pi0_aa_pt2} and \ref{fig:MR_eta_pi0_aa_z} show the $\eta$ meson mainly more suppressed than the $\pi^0$, a possible explanation can be given having in consideration that the kinematic limit $x \gtrsim 0.15$ indicates that the interaction of the virtual photon with a valence quarks dominates over the photon annihilation ($q \bar{q}$ pair production), hence the $s\bar{s}$ need it to form the $\eta$ is produced mainly out of the vacuum.

\begin{figure}[H]
\centering
\includegraphics[width=0.8\textwidth,keepaspectratio]{MR_pi0_eta_Pt2.pdf}
\caption{Multiplicity ratio comparison between $\eta(\gamma\gamma)$ and $\pi^0$, $p_T^2$ dependency.}
\label{fig:MR_eta_pi0_aa_pt2}
\end{figure} 
%
\section{Comparison between $\pi^0$ and $\eta$}
\begin{figure}[H]
\centering
\includegraphics[width=0.8\textwidth,keepaspectratio]{MR_pi0_eta_Z.pdf}
\caption{Multiplicity ratio comparison between $\eta(\gamma\gamma)$ and $\pi^0$, z dependency.}
\label{fig:MR_eta_pi0_aa_z}
\end{figure} 
% 
%
%CONCLUSIONNNNNNNNNNNNNNNN
%
%
The low statistic makes the results strongly dependent on the background model a multi-fold binning improves the determination of the background model, nevertheless here it was not possible to do such multi-fold binning. The most stable result was obtained with a simple cut on the invariant mass spectrum of the decaying products.

The results obtained with the background subtraction are in the appendix \ref{sec:res_bkg_sub} and the fitting results in each bin in \ref{sec:fitting_pics}.

%
\subsection{Multiplicity Ratio $\eta \rightarrow \pi^+\pi^-\pi^0$ decay}
The identification of the eta meson is done by making a cut on the invariant mass spectrum.

%
\begin{figure}[H]
\centering
\includegraphics[scale=0.3]{pic_3pi/hNh_NuC.gif}
\caption{Number of $\eta$ as a function of $\nu$ C-D setup. Full dot: event from solid target, Open squared: event from Deuterium.}
\label{fig:Nh_eta_3pi_Nu_C}
\end{figure}

\begin{figure}[H]
\centering
\includegraphics[scale=0.3]{pic_3pi/hNh_NuFe.gif}
\caption{Number of $\eta$ as a function of $\nu$ Fe-D setup. Full dot: event from solid target, Open squared: event from Deuterium.}
\label{fig:Nh_eta_3pi_Nu_Fe}
\end{figure}

\begin{figure}[H]
\centering
\includegraphics[scale=0.3]{pic_3pi/hNh_NuPb.gif}
\caption{Number of $\eta$ as a function of $\nu$ Pb-D setup. Full dot: event from solid target, Open squared: event from Deuterium.}
\label{fig:Nh_eta_3pi_Nu_Pb}
\end{figure}

\begin{figure}[H]
\centering
\includegraphics[scale=0.3]{pic_3pi/hNh_Pt2C.gif}
\caption{Number of $\eta$ as a function of $p_T^2$ C-D setup. Full dot: event from solid target, Open squared: event from Deuterium.}
\label{fig:Nh_eta_3pi_Pt2_C}
\end{figure}

\begin{figure}[H]
\centering
\includegraphics[scale=0.3]{pic_3pi/hNh_Pt2Fe.gif}
\caption{Number of $\eta$ as a function of $p_T^2$ Fe-D setup. Full dot: event from solid target, Open squared: event from Deuterium.}
\label{fig:Nh_eta_3pi_Pt2_Fe}
\end{figure}

\begin{figure}[H]
\centering
\includegraphics[scale=0.3]{pic_3pi/hNh_Pt2Pb.gif}
\caption{Number of $\eta$ as a function of $p_T^2$ Pb-D setup. Full dot: event from solid target, Open squared: event from Deuterium.}
\label{fig:Nh_eta_3pi_Pt2_Pb}
\end{figure}



\begin{figure}[H]
\centering
\includegraphics[scale=0.3]{pic_3pi/hNh_Q2C.gif}
\caption{Number of $\eta$ as a function of $Q^2$ C-D setup. Full dot: event from solid target, Open squared: event from Deuterium.}
\label{fig:Nh_eta_3pi_Q2_C}
\end{figure}

\begin{figure}[H]
\centering
\includegraphics[scale=0.3]{pic_3pi/hNh_Q2Fe.gif}
\caption{Number of $\eta$ as a function of $Q^2$ Fe-D setup. Full dot: event from solid target, Open squared: event from Deuterium.}
\label{fig:Nh_eta_3pi_Q2_Fe}
\end{figure}

\begin{figure}[H]
\centering
\includegraphics[scale=0.3]{pic_3pi/hNh_Q2Pb.gif}
\caption{Number of $\eta$ as a function of $Q^2$ Pb-D setup. Full dot: event from solid target, Open squared: event from Deuterium.}
\label{fig:Nh_eta_3pi_Q2_Pb}
\end{figure}


\begin{figure}[H]
\centering
\includegraphics[scale=0.3]{pic_3pi/hNh_ZC.gif}
\caption{Number of $\eta$ as a function of $z$ C-D setup. Full dot: event from solid target, Open squared: event from Deuterium.}
\label{fig:Nh_eta_3pi_Z_C}
\end{figure}
%
\begin{figure}[H]
\centering
\includegraphics[scale=0.3]{pic_3pi/hNh_ZFe.gif}
\caption{Number of $\eta$ as a function of $z$ Fe-D setup. Full dot: event from solid target, Open squared: event from Deuterium.}
\label{fig:Nh_eta_3pi_Z_Fe}
\end{figure}
%
\begin{figure}[H]
\centering
\includegraphics[scale=0.3]{pic_3pi/hNh_ZPb.gif}
\caption{Number of $\eta$ as a function of $z$ Pb-D setup. Full dot: event from solid target, Open squared: event from Deuterium.}
\label{fig:Nh_eta_3pi_Z_Pb}
\end{figure}
%
\begin{figure}[H]
\centering
\includegraphics[scale=0.3]{pic_3pi/R_Nu_3pi.gif}
\caption{Multiplicity Ratio as a function of $\nu$. Red: Carbon, Blue: Iron, Black: Lead.}
\label{fig:MR_eta_3pi_Nu}
\end{figure}
%
\begin{figure}[H]
\centering
\includegraphics[scale=0.3]{pic_3pi/R_Pt2_3pi.gif}
\caption{Multiplicity Ratio as a function of $p_T^2$. Red: Carbon, Blue: Iron, Black: Lead.}
\label{fig:MR_eta_3pi_Pt2}
\end{figure}
%
\begin{figure}[H]
\centering
\includegraphics[scale=0.3]{pic_3pi/R_Q2_3pi.gif}
\caption{Multiplicity Ratio as a function of $Q^2$. Red: Carbon, Blue: Iron, Black: Lead.}
\label{fig:MR_eta_3pi_Q2}
\end{figure}
%
\begin{figure}[H]
\centering
\includegraphics[scale=0.3]{pic_3pi/R_Z_3pi.gif}
\caption{Multiplicity Ratio as a function of $z$. Red: Carbon, Blue: Iron, Black: Lead.}
\label{fig:MR_eta_3pi_z}
\end{figure}
%
\section{Comparison of two different decay channels}
The multiplicity ratio comparison was performed using the data without acceptance correction.
%
\begin{figure}[H]
\centering
\includegraphics[scale=0.3]{MR_z_aa_3pi_cmp.gif}
\caption{Multiplicity Ratio comparison as a function of $z$.}
\label{fig:MR_eta_cmp_z}
\end{figure}
%
\begin{figure}[H]
\centering
\includegraphics[scale=0.3]{MR_Pt2_aa_3pi_cmp.gif}
\caption{Multiplicity Ratio comparison as a function of $p_T^2$.}
\label{fig:MR_eta_cmp_pt2}
\end{figure}

\section{Radiative Correction on eta}
The estimation of the radiative correction was carried out using the data from $\eta \rightarrow \gamma\gamma$. There was no binning except for $\phi_{pq}$, where 12 equispaced bins between $-180^o$ and $180^o$ where used. The value at the centroid of the distribution was used for the unbinned variables. 

\begin{figure}[H]
\centering
\includegraphics[scale=0.2]{PhiPQ_C.png}
\caption{Number of $\eta$ obtained as a function of $\phi_{pq}$ C-D setup. Left to right: before and after acceptance correction. Top to bottom: solid and liquid.}
\label{fig:Nh_eta_aa_phipq_C}
\end{figure}
%
\begin{figure}[H]
\centering
\includegraphics[scale=0.2]{PhiPQ_Fe.png}
\caption{Number of $\eta$ obtained as a function of $\phi_{pq}$ Fe-D setup. Left to right: before and after acceptance correction. Top to bottom: solid and liquid.}
\label{fig:Nh_eta_aa_phipq_Fe}
\end{figure}
%
\begin{figure}[H]
\centering
\includegraphics[scale=0.2]{PhiPQ_Pb.png}
\caption{Number of $\eta$ obtained as a function of $\phi_{pq}$ Pb-D setup. Left to right: before and after acceptance correction. Top to bottom: solid and liquid.}
\label{fig:Nh_eta_aa_phipq_Pb}
\end{figure}
%
\begin{figure}[H]
\centering
\includegraphics[scale=0.2]{PhiPQ_fit_C.png}
\caption{Fit obtained over data + AC C-D setup. Top to bottom: solid and liquid.}
\label{fig:fit_eta_aa_phipq_C}
\end{figure}
%
\begin{figure}[H]
\centering
\includegraphics[scale=0.2]{PhiPQ_fit_Fe.png}
\caption{Fit obtained over data + AC Fe-D setup. Top to bottom: solid and liquid.}
\label{fig:fit_eta_aa_phipq_Fe}
\end{figure}
%
\begin{figure}[H]
\centering
\includegraphics[scale=0.2]{PhiPQ_fit_Pb.png}
\caption{Fit obtained over data + AC Pb-D setup. Top to bottom: solid and liquid.}
\label{fig:fit_eta_aa_phipq_Pb}
\end{figure}


\begin{figure}[H]
\centering
\includegraphics[scale=0.3]{RC_eta_C.gif}
\caption{Radiative correction factor $\delta_{RC}$ for C.}
\label{fig:RC_eta_C}
\end{figure}


\begin{figure}[H]
\centering
\includegraphics[scale=0.3]{RC_eta_Fe.gif}
\caption{Radiative correction factor $\delta_{RC}$ for Fe.}
\label{fig:RC_eta_Fe}
\end{figure}


\begin{figure}[H]
\centering
\includegraphics[scale=0.3]{RC_eta_Pb.gif}
\caption{Radiative correction factor $\delta_{RC}$ for Pb.}
\label{fig:RC_eta_Pb}
\end{figure}

The results obtained for $\pi^0$ and its evolution over the bins are shown on Fig. \ref{fig:RC_pi0_hist} and \ref{fig:RC_pi0_bin}. The values of the fit parameters used $(A,A_{c},A_{cc})$ corresponds to the values obtained by Taisiya Mineeva analysis.  

\begin{figure}[H]
\centering
\includegraphics[scale=0.6]{delta_hists.pdf}
\caption{$\pi^0$ radiative correction obtained for all targets.}
\label{fig:RC_pi0_hist}
\end{figure}

\begin{figure}[H]
\centering
\includegraphics[scale=0.6]{Deltas.pdf}
\caption{$\pi^0$ radiative correction factors evolution with bins.}
\label{fig:RC_pi0_bin}
\end{figure}

The electron radiative correction factor obtained for Lead is shown on picture \ref{fig:RC_e}.  The corrections obtained for the other nuclei are less than the one obtained for Lead.

\begin{figure}[H]
\centering
\includegraphics[scale=0.3]{rad_corr_x_nu.gif}
\caption{Electron radiative correction factors for Pb.}
\label{fig:RC_e}
\end{figure}
 