\chapter{Appendix}
\label{ch:appendix}
\section{Target Fits}
\label{sec:TargetFits}
Consider the reconstruction of the position of the vertex of an interaction.
\begin{itemize}
\item If the target were just a point, of infinitesimal volume, this position will be reconstructed as a Gaussian function, due to the big number of random variables that are being summed (central limit theorem).
\item In reality the target doesn't has differential volume, then it doesn't has differential dimensions.
\end{itemize}
The reconstructed position will be then a Gaussian convoluted with the spatial function describing the target.

Let's focus on z position. Let be $f_t(z)$ the function that describes the target spatial function, $f_m(z)$ the function that describes the procedure of measurement and $v(z)$ the reconstructed vertex function $v(z) = f_t \star f_m$.

first point:
\begin{align}
f_t(z) &= \delta (z-m) \label{tf_1}\\
f_m(z) &= \frac{1}{\sqrt{2\sigma^2}}e^{\frac{-z^2}{2\sigma^2}}\label{mf_1}
\end{align}
\begin{align}
v(z) &= \int_{-\infty}^{\infty} f_t(z-t)f_m(t)dt\label{rvf_1}\\
v(z) &= \int_{-\infty}^{\infty} \delta(z-t-m)f_m(t)dt\\
v(z) &= f_m(z-m) = \frac{1}{\sqrt{2\sigma^2}}e^{\frac{-(z-m)^2}{2\sigma^2}}\\
\end{align}

Considering the real target as a step function:
\begin{equation}
f_t(z) = \left\lbrace\begin{matrix}
1,~~~|z-m|<\frac{w}{2}\\
0,~~~ o.w.
\end{matrix}\right. \label{tf_2}
\end{equation}
using \eqref{rvf_1} and \eqref{tf_2}.
\begin{align}
v(z) &= \int_{-\infty}^{\infty} f_t(z-t)f_m(t)dt\label{rvf_2}\\
v(z) &= \int_{m-z-w/2}^{m-z+w/2} f_m(t)dt\\
v(z) &= \int_{m-z-w/2}^{m-z+w/2} \frac{1}{\sqrt{2\sigma^2}}e^{\frac{-t^2}{2\sigma^2}}dt\label{vz_2b}
\end{align}

considering the function $erf$

\begin{equation}
erf(t) = \frac{2}{\sqrt{\pi}}\int_0^t e^{-x^2}dx \label{erf}
\end{equation}
using \eqref{erf} and \eqref{vz_2b}:
%
\begin{equation}
v(z) = \frac{1}{2}\left(erf(\frac{m-z+w/2}{\sqrt{2}\sigma} ) - erf(\frac{m-z-w/2}{\sqrt{2}\sigma} )\right)
\end{equation}
A fit of this function plus a second order polynomial on Carbon and Deuterium events is shown in figures \ref{fig:D} and \ref{fig:C}.
%
\begin{figure}[!ht]
\includegraphics[width=0.9\textwidth]{CD_D_fit_pol2.gif}
\caption{Fitting on Deuterium target using C and D2 data.}
\label{fig:D}
\end{figure}
%
\begin{figure}[!hb]
\includegraphics[width=0.9\textwidth]{CD_C_fit_pol2.gif}
\caption{Fitting on Carbon target using C and D2 data.}
\label{fig:C}
\end{figure}
%
As expected, the dispersion due to vertex measurement ($\sigma$ on $f_m(z)$) are very close considering the estimations using each of the targets.\\
D2: $\sigma = 0.2093$\\
C: $\sigma= 0.2065$\\
%
The limits are based on $|z-m|<w/2 + 3\sigma$\\
Deuterium:
\begin{align}
-31.8<z<-28.4
\end{align}
Carbon:
\begin{align}
-25.33<z<-24.10
\end{align}
\section{Statistical Errors}
The errors are estimated in the following way.

Consider a quantity $Q_p$ estimated using a product of $N$ other quantities $q_k$:
\begin{equation}
Q_p = \prod_{k=1}^N q_k
\end{equation}

Making a Taylor expansion around the measurement of the $q_k$ and assuming all $q_k$ are not correlated random variables. keeping only linear terms.

\begin{equation}
\Delta Q_p = \sum_{k=1}^N \prod_{i\neq k}q_i \Delta q_k \label{eq:taylor_expansion}
\end{equation}
hence the variance is:

\begin{align}
\sigma_{Q_p}^2 &= \sum_{k=1}^N (\prod_{i\neq k}q_i)^2 \sigma_{q_k}^2 \\
\frac{\sigma_{Q_p}^2}{Q_p^2} &= \sum_{k=1}^N \frac{\sigma_{q_k}^2}{q_k^2} \label{eq:var_prod}
\end{align}
%
On the other side if we consider $Q_d$ as a division of quantities:
%
\begin{equation}
Q_d = \frac{q_a}{q_b}
\end{equation}
The expansion and variance are:
\begin{align}
\Delta Q_d &= \frac{1}{q_b}\Delta q_a - \frac{q_a}{q_b^2}\Delta q_b\\
\sigma_{Q_d}^2 &= (\frac{1}{q_b})^2\sigma_{q_a}^2 + (\frac{q_a}{q_b^2})^2\sigma_{q_b}^2\\
\frac{\sigma_{Q_d}^2}{Q_d^2} &= \frac{\sigma_{q_a}^2}{q_a^2} + \frac{\sigma_{q_b}^2}{q_b^2} \label{eq:var_div}
\end{align}
%
Using \eqref{eq:var_prod} and \eqref{eq:var_div} we can estimate the error of the multiplicity ratio:

\begin{equation}
R(b) = \frac{\overbrace{\left(N_h(b)/\epsilon(b)/N_e(Q^2,\nu)\right)_A}^u} {\underbrace{\left(N_h(b)/\epsilon(b)/N_e(Q^2,\nu)\right)_D}_d}
 \label{eq:error_MR}
\end{equation}

\begin{align}
\frac{\sigma_R^2}{R^2} &= \frac{\sigma_u^2}{u^2} + \frac{\sigma_d^2}{d^2}\\
\frac{\sigma_R^2}{R^2} &= \left(\frac{\sigma_{N_h}^2}{N_h^2} + \frac{\sigma_{\epsilon}^2}{\epsilon^2} +\frac{\sigma_{N_e}^2}{N_e^2}\right)_A + \left(\frac{\sigma_{N_h}^2}{N_h^2} + \frac{\sigma_{\epsilon}^2}{\epsilon^2} +\frac{\sigma_{N_e}^2}{N_e^2}\right)_D
\end{align}

Considering Poisson distributions \footnote{acceptance could be considered as a Binomial process, in this case the error will be $\sqrt{\frac{\epsilon(b)(1-\epsilon(b))}{N_{th}(b)}}$ } the error is:
\begin{align}
\frac{\sigma_R^2}{R^2} &= \left(\frac{N_h}{N_h^2} + \frac{N_{acc}}{N_{acc}^2} +\frac{N_{th}}{N_{th}^2} + \frac{N_e}{N_e^2}\right)_A + \left(\frac{N_h}{N_h^2} + \frac{N_{acc}}{N_{acc}^2} +\frac{N_{th}}{N_{th}^2} + \frac{N_e}{N_e^2}\right)_D \\
\sigma_R &= R\sqrt{ \left(\frac{1}{N_h} + \frac{1}{N_{acc}} +\frac{1}{N_{th}} + \frac{1}{N_e}\right)_A + \left(\frac{1}{N_h} + \frac{1}{N_{acc}} +\frac{1}{N_{th}} + \frac{1}{N_e}\right)_D}
\end{align}

To include fitting procedures and/or radiative corrections through multiplication or division by a weight, this expression is modify accordingly.

%
%\section*{Appendix}
%Let be $x$ the measurement from a Binomial process, where $N$ Bernoulli experiment were performed.
%According to Maximum Likelihood criteria the best estimator $\hat{p}$ of the parameter $p$  of the distribution is the parameter that maximize the likelihood function $\mathcal{L}$:
%\begin{equation}
%\mathcal{L}(p|x) = \mathcal{P}(x,p,N)
%\end{equation}
%where
%\begin{equation}
%\mathcal{P}(x,p,N) = \binom{N}{x}p^x(1-p)^{N-x}
%\end{equation}
%%
%$\max \mathcal{L}(p|x)$:
%\begin{align}
%\frac{d\mathcal{L}}{dp} &= \frac{d\mathcal{P}}{dp} = \binom{N}{x}\left[ xp^{x-1}(1-p)^{N-x} - (N-x)p^x(1-p)^{N-x-1} \right] =0 \nonumber \\
%&x(1-p) - (N-x)p = 0\nonumber\\
%\hat{p} &= x/N
%\end{align}
%The variance of this estimation is:
%\begin{align}
%\mathcal{E}((\hat{p} - p)^2) &= \mathcal{E}(\hat{p}^2) + \mathcal{E}(p^2) - 2\mathcal{E}(\hat{p}p)\\
%&=\frac{1}{N^2}\mathcal{E}(x^2) + p^2 -2p\frac{\mathcal{E}(x)}{N}\\
%&=\frac{1}{N^2}(Np(1-p) + (Np)^2) + p^2 -2p^2\\
%&=\frac{1}{N}(p(1-p))
%\end{align}
%Hence an estimation of the uncertainty on taking this estimator is:
%\begin{equation}
%\hat{\sigma} _{\hat{p}}^2 = \frac{x/N(1-x/N)}{N}
%\end{equation}
\section{Result with background subtraction}
\label{sec:res_bkg_sub}
The number of Eta extracted in the different z bins as well as the values corrected with the acceptance are shown in Fig. \ref{fig:Nh_eta_aa_z_C}, \ref{fig:Nh_eta_aa_z_Fe} and \ref{fig:Nh_eta_aa_z_Pb}. The Acceptance dependency on z bins are shown in Fig. \ref{fig:AC_eta_aa_z}.
%
\begin{figure}[H]
\centering
\includegraphics[scale=0.25]{Zeta_aa_Nh_C.png}
\caption{Number of $\eta$ reconstructed as a function of $z$ in C-D setup. Left and right panels correspond to before and after acceptance correction, respectively. Upper and lower panels correspond to solid and liquid, respectively.}
\label{fig:Nh_eta_aa_z_C}
\end{figure}
%
\begin{figure}[H]
\centering
\includegraphics[scale=0.25]{Zeta_aa_Nh_Fe.png}
\caption{Number of $\eta$ reconstructed as a function of $z$ in Fe-D setup. Left and right panels correspond to before and after acceptance correction, respectively.  Upper and lower panels correspond to solid and liquid, respectively.}
\label{fig::Nh_eta_aa_z_Fe}
\end{figure}
%
\begin{figure}[H]
\centering
\includegraphics[scale=0.25]{Zeta_aa_Nh_Pb.png}
\caption{Number of $\eta$ reconstructed as a function of $z$ in Pb-D setup. Left and right panels correspond to before and after acceptance correction, respectively.  Upper and lower panels correspond to solid and liquid, respectively.}
\label{fig::Nh_eta_aa_z_Pb}
\end{figure}
%
\begin{figure}[H]
\centering
\includegraphics[scale=0.25]{Zeta_aa_AC.png}
\caption{Acceptance as a function of $z$}
\label{fig:AC_eta_aa_z}
\end{figure}
%
\clearpage
The number of Eta extracted in the different $\nu$ bins as well as the values corrected with the acceptance are shown in Fig. \ref{fig:Nh_eta_aa_nu_C}, \ref{fig:Nh_eta_aa_nu_Fe} and \ref{fig:Nh_eta_aa_nu_Pb}. The Acceptance dependency on $\nu$ bins are shown in Fig. \ref{fig:AC_eta_aa_nu}.
\begin{figure}[H]
\centering
\includegraphics[scale=0.25]{Nueta_aa_Nh_C.png}
\caption{Number of $\eta$ reconstructed as a function of $\nu$ in C-D setup. Left and right panels correspond to before and after acceptance correction, respectively. Upper and lower panels correspond to solid and liquid, respectively.}
\label{fig:Nh_eta_aa_nu_C}
\end{figure}
%
\begin{figure}[H]
\centering
\includegraphics[scale=0.25]{Nueta_aa_Nh_Fe.png}
\caption{Number of $\eta$ reconstructed as a function of $\nu$ in Fe-D setup. Left and right panels correspond to before and after acceptance correction, respectively. Upper and lower panels correspond to solid and liquid, respectively.}
\label{fig::Nh_eta_aa_nu_Fe}
\end{figure}
%
\begin{figure}[H]
\centering
\includegraphics[scale=0.25]{Nueta_aa_Nh_Pb.png}
\caption{Number of $\eta$ reconstructed as a function of $\nu$ in Pb-D setup. Left and right panels correspond to before and after acceptance correction, respectively. Upper and lower panels correspond to solid and liquid, respectively.}
\label{fig::Nh_eta_aa_nu_Pb}
\end{figure}
%
\begin{figure}[H]
\centering
\includegraphics[scale=0.25]{Nueta_aa_AC.png}
\caption{Acceptance as a function of $\nu$}
\label{fig:AC_eta_aa_nu}
\end{figure}
%
\clearpage
The number of Eta extracted in the different $Q^2$ bins as well as the values corrected with the acceptance are shown in Fig. \ref{fig:Nh_eta_aa_q2_C}, \ref{fig:Nh_eta_aa_q2_Fe} and \ref{fig:Nh_eta_aa_q2_Pb}. The Acceptance dependency on $Q^2$ bins are show in Fig. \ref{fig:AC_eta_aa_q2}.
\begin{figure}[H]
\centering
\includegraphics[scale=0.25]{Q2eta_aa_Nh_C.png}
\caption{Number of $\eta$ reconstructed as a function of $Q^2$ in C-D setup. Left and right panels correspond to before and after acceptance correction, respectively. Upper and lower panels correspond to solid and liquid, respectively.}
\label{fig:Nh_eta_aa_q2_C}
\end{figure}
%
\begin{figure}[H]
\centering
\includegraphics[scale=0.25]{Q2eta_aa_Nh_Fe.png}
\caption{Number of $\eta$ reconstructed as a function of $Q^2$ in Fe-D setup. Left and right panels correspond to before and after acceptance correction, respectively. Upper and lower panels correspond to solid and liquid, respectively.}
\label{fig::Nh_eta_aa_q2_Fe}
\end{figure}
%
\begin{figure}[H]
\centering
\includegraphics[scale=0.25]{Q2eta_aa_Nh_Pb.png}
\caption{Number of $\eta$ reconstructed as a function of $Q^2$ in Pb-D setup. Left and right panels correspond to before and after acceptance correction, respectively. Upper and lower panels correspond to solid and liquid, respectively.}
\label{fig::Nh_eta_aa_q2_Pb}
\end{figure}
%
\begin{figure}[H]
\centering
\includegraphics[scale=0.25]{Q2eta_aa_AC.png}
\caption{Acceptance as a function of $Q^2$}
\label{fig:AC_eta_aa_q2}
\end{figure}
%
\clearpage
The number of Eta extracted in the different $p_T^2$ bins as well as the values corrected with the acceptance are shown in Fig. \ref{fig:Nh_eta_aa_pt2_C}, \ref{fig:Nh_eta_aa_pt2_Fe} and \ref{fig:Nh_eta_aa_pt2_Pb}. The Acceptance dependency on $p_T^2$ bins are show in Fig. \ref{fig:AC_eta_aa_pt2}.
\begin{figure}[H]
\centering
\includegraphics[scale=0.25]{Pt2eta_aa_Nh_C.png}
\caption{Number of $\eta$ reconstructed as a function of $p_T^2$ in C-D setup. Left and right panels correspond to before and after acceptance correction, respectively. Upper and lower panels correspond to solid and liquid, respectively.}
\label{fig:Nh_eta_aa_pt2_C}
\end{figure}
%
\begin{figure}[H]
\centering
\includegraphics[scale=0.25]{Pt2eta_aa_Nh_Fe.png}
\caption{Number of $\eta$ reconstructed as a function of $p_T^2$ in Fe-D setup. Left and right panels correspond to before and after acceptance correction, respectively. Upper and lower panels correspond to solid and liquid, respectively.}
\label{fig::Nh_eta_aa_pt2_Fe}
\end{figure}
%
\begin{figure}[H]
\centering
\includegraphics[scale=0.25]{Pt2eta_aa_Nh_Pb.png}
\caption{Number of $\eta$ reconstructed as a function of $p_T^2$ in Pb-D setup. Left and right panels correspond to before and after acceptance correction, respectively. Upper and lower panels correspond to solid and liquid, respectively.}
\label{fig::Nh_eta_aa_pt2_Pb}
\end{figure}
%
\begin{figure}[H]
\centering
\includegraphics[scale=0.25]{Pt2eta_aa_AC.png}
\caption{Acceptance as a function of $P_t^{2}$}
\label{fig:AC_eta_aa_pt2}
\end{figure}
%
The projection of the multiplicity ratio over the different kinematical variables are shown in Fig. \ref{fig:MR_eta_aa_z}, \ref{fig:MR_eta_aa_pt2}, \ref{fig:MR_eta_aa_nu} and \ref{fig:MR_eta_aa_q2}.
\begin{figure}[H]
\centering
\includegraphics[scale=0.25]{Zeta_aa_pics/MR_Z_cmp.gif}
\caption{Multiplicity Ratio as a function of $z$}
\label{fig:MR_eta_aa_z}
\end{figure}
%
\begin{figure}[H]
\centering
\includegraphics[scale=0.25]{Pt2eta_aa_pics/MR_Pt2_cmp.gif}
\caption{Multiplicity Ratio as a function of $p_T^2$}
\label{fig:MR_eta_aa_pt2}
\end{figure}
%
\begin{figure}[H]
\centering
\includegraphics[scale=0.25]{Nueta_aa_pics/MR_Nu_cmp.gif}
\caption{Multiplicity Ratio as a function of $\nu$}
\label{fig:MR_eta_aa_nu}
\end{figure}
%
\begin{figure}[H]
\centering
\includegraphics[scale=0.25]{Q2eta_aa_pics/MR_Q2_cmp.gif}
\caption{Multiplicity Ratio as a function of $Q^2$}
\label{fig:MR_eta_aa_q2}
\end{figure}

\section{Fitting pictures}
\label{sec:fitting_pics}
The pictures below are the result of the fitting procedure. A proper initialization of the model parameters (\eqref{eq:eta_aa_model}) was done for every bin.
 
\begin{landscape}
\begin{figure}[H]
\centering
\includegraphics[scale=0.25]{eta_aa_q2_fit_C.png}
\caption{Fitting procedure on $Q^2$ bins. C data.}
\label{fig:eta_aa_q2_fit_C}
\end{figure}

%
\begin{figure}[H]
\centering
\includegraphics[scale=0.25]{eta_aa_q2_fit_Fe.png}
\caption{Fitting procedure on $Q^2$ bins. Fe data.}
\label{fig:eta_aa_q2_fit_Fe}
\end{figure}
%
\begin{figure}[H]
\centering
\includegraphics[scale=0.25]{eta_aa_q2_fit_Pb.png}
\caption{Fitting procedure on $Q^2$ bins. Pb data.}
\label{fig:eta_aa_q2_fit_Pb}
\end{figure}
%
\begin{figure}[H]
\centering
\includegraphics[scale=0.23]{eta_aa_q2_fit_C_sim.png}
\caption{Fitting procedure on $Q^2$ bins. Simulation reconstructed.}
\label{fig:eta_aa_q2_fit_sim}
\end{figure}

\begin{figure}[H]
\centering
\includegraphics[scale=0.23]{eta_aa_nu_fit_C.png}
\caption{Fitting procedure on $\nu$ bins. C data.}
\label{fig:eta_aa_nu_fit_C}
\end{figure}

%
\begin{figure}[H]
\centering
\includegraphics[scale=0.23]{eta_aa_nu_fit_Fe.png}
\caption{Fitting procedure on $\nu$ bins. Fe data.}
\label{fig:eta_aa_nu_fit_Fe}
\end{figure}
%
\begin{figure}[H]
\centering
\includegraphics[scale=0.23]{eta_aa_nu_fit_Pb.png}
\caption{Fitting procedure on $\nu$ bins. Pb data.}
\label{fig:eta_aa_nu_fit_Pb}
\end{figure}
%
\begin{figure}[H]
\centering
\includegraphics[scale=0.23]{eta_aa_nu_fit_C_sim.png}
\caption{Fitting procedure on $\nu$ bins. Simulation reconstructed.}
\label{fig:eta_aa_nu_fit_sim}
\end{figure}


\begin{figure}[H]
\centering
\includegraphics[scale=0.23]{eta_aa_z_fit_C.png}
\caption{Fitting procedure on $z$ bins. C data.}
\label{fig:eta_aa_z_fit_C}
\end{figure}

%
\begin{figure}[H]
\centering
\includegraphics[scale=0.23]{eta_aa_z_fit_Fe.png}
\caption{Fitting procedure on $z$ bins. Fe data.}
\label{fig:eta_aa_z_fit_Fe}
\end{figure}
%
\begin{figure}[H]
\centering
\includegraphics[scale=0.23]{eta_aa_z_fit_Pb.png}
\caption{Fitting procedure on $z$ bins. Pb data.}
\label{fig:eta_aa_z_fit_Pb}
\end{figure}
%
\begin{figure}[H]
\centering
\includegraphics[scale=0.23]{eta_aa_z_fit_C_sim.png}
\caption{Fitting procedure on $z$ bins. Simulation reconstructed.}
\label{fig:eta_aa_z_fit_sim}
\end{figure}


\begin{figure}[H]
\centering
\includegraphics[scale=0.23]{eta_aa_pt2_fit_C.png}
\caption{Fitting procedure on $P_t^2$ bins. C data.}
\label{fig:eta_aa_pt2_fit_C}
\end{figure}

%
\begin{figure}[H]
\centering
\includegraphics[scale=0.23]{eta_aa_pt2_fit_Fe.png}
\caption{Fitting procedure on $P_t^2$ bins. Fe data.}
\label{fig:eta_aa_pt2_fit_Fe}
\end{figure}
%
\begin{figure}[H]
\centering
\includegraphics[scale=0.23]{eta_aa_pt2_fit_Pb.png}
\caption{Fitting procedure on $P_t^2$ bins. Pb data.}
\label{fig:eta_aa_pt2_fit_Pb}
\end{figure}
%
\begin{figure}[H]
\centering
\includegraphics[scale=0.23]{eta_aa_pt2_fit_C_sim.png}
\caption{Fitting procedure on $P_t^2$ bins. Simulation reconstructed.}
\label{fig:eta_aa_pt2_fit_sim}
\end{figure}


\end{landscape}
