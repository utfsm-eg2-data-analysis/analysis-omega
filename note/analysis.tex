\chapter{Data Analysis}
\label{ch:data_ana}
The cooking used in this analysis corresponds to the "pass2" \footnote{/mss/clas/eg2a/production/Pass2/Clas} and was performed by Taisiya Mineeva.

To perform the PID, a c++ class called \textbf{Analyser} was used. It contains implementations of all the cuts used in this analysis. This class was first developed by Hayk Hakobyan and later expanded by myself. It allows the easy reuse of this work by anyone who is interested in CLAS data analysis. The class make use of the standard SEB (simple event builder) distribution in the ClasTool format. The details of the bank definitions are in \cite{CLAS_BANKS}
%
\clearpage
\section{Event selection}
The event topology considered in this analysis are:
\begin{align}
ep&\rightarrow e\gamma\gamma X \\
ep&\rightarrow e\pi^+\pi^-\gamma\gamma X
\end{align}
\subsection{Electron identification}
\label{s:e_id}
The particles in the EVNT bank are ordered according to their arrival time to SC. Some of the particles will leave tracks in all detectors, the others only in part of them. The signs of the  electrical charges of the particles are extracted from the DC by the shape and inclination direction of the bending produced by the toroidal field.
We require that the first identified particle must be an electron, hence the first row of the EVNT bank must fulfil the electron requirement. The basic requirements for the electron candidate are:
\begin{itemize}
\item The particle must leave a track in all detectors (DC,CC,SC,EC).
\item The charge must be negative.
\end{itemize}
In terms of Bank information it is expressed as:
\begin{itemize}
\item 0 < EVNT.status < 100 \\
	rows in CCPB != 0\\
	rows in ECPB != 0\\
	rows in SCPB != 0\\
	EVNT.CCStat > 0 \\
	EVNT.SCStat > 0 \\
	EVNT.DCStat > 0 \\
	EVNT.ECStat > 0
\item EVNT.Charge == -1
\end{itemize}
The next step is to make $\pi^-/e^-$ separation using the information in the Cherenkov counter (CCPB), to do so a minimum number of photo-electrons is required in each sector of the CC.

\begin{figure}[!ht]
\centering
\includegraphics[scale=0.15]{Nphe_e.png}
\caption{Number of photo-electrons threshold on CC. Red filled area correspond to electron selection.}
\label{fig:Nphe_e}
\end{figure}

After this cuts, additional cuts are necessary in order to assure the quality of the electrons. A set of fiducial cuts clean the data from reconstruction inefficiencies near the edges of the detectors. In this analysis two fiducial cuts on electrons are used, one on DC \cite{Lorenzo_analysis_note} and the other on EC \eqref{eq:EC_efid}, see Fig. \ref{fig:fid_e_EC}
\begin{equation}
40<U<400,V<360,W<390~~cm
\label{eq:EC_efid}
\end{equation}
\begin{figure}[!ht]
\centering
\includegraphics[scale=0.3]{fid_DC.png}
\caption{$\theta_{lab}$ vs $\phi_{lab}$ for electron candidates. Upper plot before fid cuts (EC + DC); lower plot after fid cuts (EC+DC)}
\label{fig:fid_e_DC}
\end{figure}
%
\begin{figure}[!ht]
\centering
\includegraphics[scale=0.3]{fid_EC.png}
\caption{Electron candidates on xy plane on EC coordinate system. Upper plot before fiducial cuts (EC + DC); lower plot after fiducial cuts (EC+DC)}
\label{fig:fid_e_EC}
\end{figure}
%
Besides the fiducial cut, the electrons are required to follow a specific sampling fraction functionality depending on the momentum. The cut used was developed by Taisiya Mineeva \cite{Taya_analysis_note} and is shown in the Fig. \ref{fig:sf_e_C}, \ref{fig:sf_e_Fe} and \ref{fig:sf_e_Pb}. The cut applied is:
\begin{equation}
\lvert E/p - E_{m} \rvert < 2.5 \sigma_{m}
\end{equation}
with  $E_{m}$ and $\sigma_{m}$ defined as:
\begin{align}
E_{m} &=  c_0(s) + c_1(s)P_e + c_1(s)P_e^2\\
\sigma_m &= \sqrt{b_0(s)^2 + b_1(s)^2/P_e}
\end{align}
The $s$ reflecting the sector dependency. The values of the parameters are shown in tables: \ref{tb:sf_Fe_m}, \ref{tb:sf_Fe_s}, \ref{tb:sf_C_m}, \ref{tb:sf_C_s}, \ref{tb:sf_Pb_m}, \ref{tb:sf_Pb_s}\\

\begin{table}[!ht]
\centering
\begin{tabular}{|r|r|r|r|}
\multicolumn{4}{c}{Iron $E_{m}$ parameters} \\ \hline
$s$ &$c_0$ & $c_1$ & $c_2$\\ \hline
0 & 0.222404 & 0.0222688 & -0.0024153\\ \hline
1 & 0.234623 & 0.0194985 & -0.00208357\\ \hline
2 & 0.252287 & 0.024248 & -0.00338846\\ \hline
3 & 0.250946 & 0.0208409 & -0.00326824\\ \hline
4 & 0.271956 & 0.0118487 & -0.00187084\\ \hline
5 & 0.252613 & 0.022819 & -0.00311242 \\ \hline
\end{tabular}
\caption{Parameters extracted from fit on Iron data. \cite{Taya_thesis}}
\label{tb:sf_Fe_m}
\end{table}

\begin{table}[!ht]
\centering
\begin{tabular}{|r|r|r|}
\multicolumn{3}{c}{Iron $\sigma_{m}$ parameters} \\ \hline
$s$ &$b_0$ & $b_1$ \\ \hline
0 & 9.23027e-03 & 2.98343e-02\\ \hline
1 & 8.66367e-03 & 3.08858e-02\\ \hline
2 & 1.07826e-02 & 2.63854e-02\\ \hline
3 & 7.22581e-03 & 2.98809e-02\\ \hline
4 & 1.84073e-02 & 3.48029e-02\\ \hline
5 & 4.11461e-03 & 3.55081e-02\\ \hline
\end{tabular}
\caption{Parameters extracted from fit on Iron data. \cite{Taya_thesis}}
\label{tb:sf_Fe_s}
\end{table}
%
\begin{table}[!ht]
\centering
\begin{tabular}{|r|r|r|r|}
\multicolumn{4}{c}{Carbon $E_{m}$ parameters} \\ \hline
$s$ &$c_0$ & $c_1$ & $c_2$\\ \hline \hline
0 & 0.252164 & 0.0122263 & -0.000793937\\ \hline
1 & 0.278574 & 0.0187482 & -0.00238217\\ \hline
2 & 0.262079 & 0.0230685 & -0.00354741\\ \hline
3 & 0.251108 & 0.0201568 & -0.00332367\\ \hline
4 & 0.263396 & 0.00955238 & -0.00102038\\ \hline
5 & 0.255245 & 0.0232659 & -0.00304798\\ \hline
\end{tabular}
\caption{Parameters extracted from fit on Carbon data. \cite{Taya_thesis}}
\label{tb:sf_C_m}
\end{table}
%
\begin{table}[!ht]
\centering
\begin{tabular}{|r|r|r|}
\multicolumn{3}{c}{Carbon $\sigma_{m}$ parameters} \\ \hline
$s$ &$b_0$ & $b_1$ \\ \hline
0 & 9.55113e-03 & 3.40672e-02\\ \hline
1 & 1.39889e-02 & 3.74682e-02\\ \hline
2 & 9.32762e-03 & 2.90046e-02\\ \hline
3 & 8.21055e-03 & 2.98893e-02\\ \hline
4 & 2.25684e-02 & 3.06508e-02\\ \hline
5 & 1.17254e-02 & 3.64221e-02\\ \hline
\end{tabular}
\caption{Parameters extracted from fit on Carbon data. \cite{Taya_thesis}}
\label{tb:sf_C_s}
\end{table}
%
\begin{table}[!ht]
\centering
\begin{tabular}{|r|r|r|r|}
\multicolumn{4}{c}{Lead $E_{m}$ parameters} \\ \hline
$s$ &$c_0$ & $c_1$ & $c_2$\\ \hline \hline
0 & 0.253431 & 0.0138251 & -0.0014016\\ \hline
1 & 0.249059 & 0.0147784 & -0.00148693\\ \hline
2 & 0.254573 & 0.022589 & -0.00305686\\ \hline
3 & 0.255589 & 0.0190419 & -0.00305263\\ \hline
4 & 0.276739 & 0.0111585 & -0.00175784\\ \hline
5 & 0.262587 & 0.0191659 & -0.0026264\\ \hline
\end{tabular}
\caption{Parameters extracted from fit on Lead data. \cite{Taya_thesis}}
\label{tb:sf_Pb_m}
\end{table}
%
\begin{table}[!ht]
\centering
\begin{tabular}{|r|r|r|}
\multicolumn{3}{c}{Lead $\sigma_{m}$ parameters} \\ \hline
$s$ &$b_0$ & $b_1$ \\ \hline
0 & 7.67408e-03 & 3.54391e-02\\ \hline
1 & 7.52798e-03 & 3.38371e-02\\ \hline
2 & 8.13241e-03 & 2.77300e-02\\ \hline
3 & 7.20303e-03 & 3.03627e-02\\ \hline
4 & 1.80841e-02 & 3.53020e-02\\ \hline
5 & 1.99220e-03 & 3.76172e-02\\ \hline
\end{tabular}
\caption{Parameters extracted from fit on Lead data. \cite{Taya_thesis}}
\label{tb:sf_Pb_s}
\end{table}

\begin{figure}[!ht]
\centering
\includegraphics[scale=0.2]{SF_cut_C.png}
\caption{Sampling fraction cut for Carbon. The events between the curves are accepted. The momentum magnitude is in GeV. Two versions of this cuts were compared.}
\label{fig:sf_e_C}
\end{figure}
%
\begin{figure}[!ht]
\centering
\includegraphics[scale=0.2]{SF_cut_Fe.png}
\caption{Sampling fraction cut for Iron. The events between the curves are accepted. The momentum magnitude is in GeV. Two versions of the cuts were compared.}
\label{fig:sf_e_Fe}
\end{figure}
%
\begin{figure}[!ht]
\centering
\includegraphics[scale=0.2]{SF_cut_Pb.png}
\caption{Sampling fraction cut for Lead. The events between the curves are accepted. The momentum magnitude is in GeV. Two versions of the cuts were compared.}
\label{fig:sf_e_Pb}
\end{figure}
%
Once the electron is identified, the next step is to select events with kinematics in the DIS regime. According to the wave length of the virtual photon $\lambda \approx 1/Q$, the selection of $Q^2>1.0$ allows to resolve partons inside the nucleons. To exclude the hadrons  coming from the decay of nuclear resonances (e.g. $\Delta ^{++}$) a cut on the invariant mass is applied: $W>2.0$. Besides that, one of the possible sources of the data contamination are the  radiative effects which are becoming more important for larger values of the DIS variable $y$, hence a cut $y<0.85$ is also applied. The phase space for the scattered electron is shown in Fig. \ref{fig:e_PS}.
%   
\begin{figure}[!ht]
\centering
\includegraphics[scale=0.6]{Electron_PS_info.png}
\caption{Electron phase space after all cuts applied.}
\label{fig:e_PS}
\end{figure}
%
Having the electron phase space defined, the next step is to identify the hadrons. 
\clearpage
\section{Eta reconstruction}
The $\eta$ meson particle main decays are in table \ref{tb:eta_BR}
%
\begin{table}
\centering
\begin{tabular}{|l|r|}
\hline
Decay channel & BR ($\Gamma_{ch}/\Gamma_{tot}$)\\ \hline
$\eta \rightarrow \gamma\gamma$ & 39.41\%\\ \hline
$\eta \rightarrow \pi^0\pi^0\pi^0$ & 32.68\%\\ \hline
$\eta \rightarrow \pi^+\pi^-\pi^0$ & 29.92 \% \\ \hline
\end{tabular}
\caption{Main $\eta$ decay channels.}
\label{tb:eta_BR}
\end{table}
%

\subsection{Photon identification}
The photons will be used to reconstruct the $\pi^0$ meson for the $\eta \rightarrow \pi^+\pi^-\pi^0$ decay and $\eta \rightarrow \gamma\gamma$.

Unlike charged particles, the photon will leave almost no track inside of the DC and no track at all inside of the CC, since CC PMTs measures light in the range $\sim 400 - 700 nm$ which is well below from what corresponds to the photons produced from the decay (< 1.973 fm) and the reflectivity of the mirrors are zero at this energies (> 100 MeV). Thus photons leave tracks only inside the EC.
The sampling fraction energy for $\gamma$ is 0.272 and it is a parameter of the reconstruction algorithm, see Fig. \ref{fig:gamma_sf}.
%
\begin{figure}[!ht]
\centering
\includegraphics[scale=0.6]{gamma_sampling_fraction.pdf}
\caption{Energy sampling fraction (E/P) included in the reconstruction algorithm.}
\label{fig:gamma_sf}
\end{figure}
%
\FloatBarrier
The ECPB has three measured energy: the energy deposited in the Inner stack, ECPB.Ein; the energy deposited in the outer stack, ECPB.Eout; and the sum of the inner and outer stack if  certain criteria is met,  ECPB.Etot. Here the measured energy of the $\gamma$ is extracted from ECPB and is defined as: $$E = \max \{ECPB.Etot,ECPB.Ein+ECPB.Eout/0.272\}$$.

There is no independent measurement of the momentum for the $\gamma$. The direction is extracted from a vector constructed from the reconstructed electron vertex of the event and the $\gamma$ hit on the EC.
A cut related to the velocity of $\gamma$ is applied: the $\gamma$ candidate must travel at the velocity of light independently of its energy, see Fig. \ref{fig:gamma_v_E}.
%
\begin{figure}[!ht]
\centering
\includegraphics[width=0.7\textwidth]{Dt_gamma_candidate_Pb.gif}
\caption{$\Delta t$ (Eq. \eqref{eq:Dt_gamma} ) of $\gamma$ candidates as a function of Energy ($E = E_{ec}/0.272$).}
\label{fig:gamma_v_E}
\end{figure}
%
\FloatBarrier
In terms of bank information this correspond to:
\begin{equation}
\begin{matrix*}[l]
ECPB.Path/(EVNT.Betta*30) - ECPB.Path/30>-2.2\\
ECPB.Path/(EVNT.Betta*30) - ECPB.Path/30<1.3
\end{matrix*}
\label{eq:Dt_gamma}
\end{equation}
%              
Finally a minimum energy of 100 MeV is required and fiducial cuts are applied to avoid EC edges, in terms of the standard UVW coordinates of the EC: 
\begin{equation}
40<U<410,0<V<370,0<W<410~~cm
\end{equation}
%
The momentum of the $\gamma$ are available in the EVNT bank. Nevertheless using the information on the ECPB the momentum can be estimated. Both versions of the momentum are not equivalent, and give rise to different invariant mass spectra.
The momentum definitions are:

From EVNT:
\begin{equation}
\begin{matrix}
P_x=P \cdot cx\\
P_y=P \cdot cy\\
P_z=P \cdot cz
\end{matrix}
\label{eq:P_EVNT}
\end{equation}
%
From ECPB:
\begin{equation}
\begin{matrix*}[l]
R_t = \sqrt{X_{EC}^2 + Y_{EC}^2}\\
R  =  \sqrt{X_{EC}^2 + Y_{EC}^2 + (Z_{EC}-vz_e)^2}~//~vz_e\textit{: electron z vertex}\\
\phi = tan^{-1}(Y_{EC}/X_{EC})~//\textit{using atan2()}\\
\theta =sin^{-1}(\frac{R_t}{R})\\
P_x=E \cdot sin(\theta)cos(\phi)\\
P_y=E \cdot sin(\theta)sin(\phi)\\
P_z=E \cdot cos(\theta)
\end{matrix*}
\label{eq:P_ECPB}
\end{equation}
%
With $X_{EC}$, $Y_{EC}$ and $Z_{EC}$ being the position of the hit over the EC. The spectrum of the $Z_{EC}$  is shown in Fig. \ref{fig:EC_Z}.
%
\begin{figure}[!ht]
\centering
\includegraphics[scale=0.3]{Z_EC.gif}
\caption{$Z_{EC}$ spectrum in centimetres from the CLAS center.}
\label{fig:EC_Z}
\end{figure}
%
The discrepancy can be seen in Fig. \ref{fig:P_ECPB_EVNT}.
%
\begin{figure}[!ht]
\centering
\includegraphics[scale=0.3]{P_ECPB_EVNT.png}
\caption{Comparison between momentum definition. On x axis extracted from ECPB, on y axis extracted from EVNT.}
\label{fig:P_ECPB_EVNT}
\end{figure}
%
The definition used here is the one extracted from ECPB, because this allows to applied momentum corrections based on the electron vertex correction.

The $\pi^0$ and $\eta$ mesons PID is done using events with at least two photons identified. Then, a combination of all possibles pairs in the event is carried out. Every pair is a meson candidate candidate ($2\gamma \rightarrow 1\eta, 3\gamma \rightarrow 3\eta,... n\gamma \rightarrow n(n-1)/2 \eta$).

Once the pairs are obtained the invariant mass spectrum is used to identify $\pi^0$ and $\eta$. The invariant mass is calculated as follows:
\begin{align}
M_{\gamma\gamma} = \sqrt{E_{\gamma_1}E_{\gamma_2}(1-cos(\theta_{\gamma\gamma}))}\\
cos(\theta_{\gamma\gamma}) = \frac{\vec{p_1}\cdot\vec{p_2}}{E_{\gamma_1}E_{\gamma_2}} \notag
\label{eq:M_aa}
\end{align}
%
\clearpage
\section{Two gamma channel}
The identification of the two gamma channel is identical to the $\pi^0$ analysis \cite{Taya_analysis_note} but the binning is one dimensional only. Five bins were used for each variable, only one variable is binned at a time, the others are constrained to their maximum and minimum. 

The absolute limits of the four dimensional phase-space are :
\begin{equation}
\begin{matrix*}[r]
1.0<Q^{2}<4.1\\
2.2<\nu<4.25\\
0.5<z<1.0\\
0<p_{T}^{2}<1.5
\end{matrix*}
\label{eq:abs_lim}
\end{equation}
Around 3.6\% of all photon pairs lie in the $\eta$ candidate mass range $0.4<M<0.72$, See Fig. \ref{fig:M_eta_aa_stat}.
%
\begin{figure}[!ht]
\centering
\includegraphics[width=0.8\textwidth,keepaspectratio]{M_eta_aa_stat.gif}
\caption{Black line correspond to all photon pairs. Blue line correspond to photon pairs within the kinematical limits. Red fill area correspond to $\eta$ candidates ($0.4<M<0.72$).}
\label{fig:M_eta_aa_stat}
\end{figure}
%

The $\eta$ candidate phase-space is binned according to table \ref{tb:eta_bins}.
\begin{table}[!ht]
\centering
\begin{tabular}{|c|r|r|r|r|r|r|}
\hline
$z$ &0.5& 0.6& 0.7& 0.8 &0.9& 1.0\\ \hline
$p_T^2$ &0 &0.18 &0.36 &0.54 &0.72 &1.5\\ \hline
$\nu$ &2.2 &2.86 &3.22 &3.58 &3.87& 4.25\\ \hline
$Q^2$ &1.0& 1.33 &1.66& 2.0 &2.5 &4.1\\ \hline
\end{tabular}
\caption{Binning used in $\eta \rightarrow \gamma\gamma$. Only one of the variables is binned at a time, the other variables are kept within their maximum and minimum values.}
\label{tb:eta_bins}
\end{table}
%
\begin{figure}[!ht]
\centering
\begin{subfigure}[b]{0.45\textwidth}
\includegraphics[width=\textwidth]{Q2_eta_aa.gif}
\caption{}
\label{fig:Q2_aa_bins}
\end{subfigure}
~
\begin{subfigure}[b]{0.45\textwidth}
\includegraphics[width=\textwidth]{Nu_eta_aa.gif}
\caption{}
\label{fig:Nu_aa_bins}
\end{subfigure}
~
\begin{subfigure}[b]{0.45\textwidth}
\includegraphics[width=\textwidth]{Z_eta_aa.gif}
\caption{}
\label{fig:Z_aa_bins}
\end{subfigure}
~
\begin{subfigure}[b]{0.45\textwidth}
\includegraphics[width=\textwidth]{Pt2_eta_aa.gif}
\caption{}
\label{fig:Pt2_aa_bins}
\end{subfigure}
\caption{All spectra are obtained after considering the limits \eqref{eq:abs_lim} and the mass cut in Fig. \ref{fig:M_eta_aa_stat}. \protect\subref{fig:Q2_aa_bins} Q2 binning. \protect\subref{fig:Nu_aa_bins} $\nu$ binning. \protect\subref{fig:Z_aa_bins} z binning. \protect\subref{fig:Pt2_aa_bins} $p_T^2$ binning.}
\end{figure}
%
The $\eta$ mesons are extracted using a model fitted to the data, the model is:
\begin{equation}
f(x) = N_{\eta}G(x;\mu,\sigma) + N_b pol2(x;a_1,a_2)
\label{eq:eta_aa_model}
\end{equation}
Where $G(x;\mu,\sigma)$ is a standard Gaussian and $pol2(x;a_1,a_2)$ is a 2nd-degree polynomial, an example of fit result is shown in Fig. \ref{fig:eta_aa_fit}.

\begin{figure}[!ht]
\centering
\includegraphics[scale=0.3]{eta_aa_fit.gif}
\caption{Upper plot: invariant mass spectrum of $2\gamma$, the fitted function corresponds to \eqref{eq:eta_aa_model}. Lower plot: plot of $(data(x)-model(x))/data_{error}(x)$ shows the fit performance.}
\label{fig:eta_aa_fit}
\end{figure}
%
Besides this bin selection, a cut on the z vertex of the electrons is applied.  The cut is applied on a corrected vertex \cite{Taya_analysis_note}, this correction takes into account the electron beam offset from the ideal beam position on the xy plane.
%
\begin{figure}[!ht]
\centering
\includegraphics[scale=0.3]{vzcut_taya.gif}
\caption{Electron z vertex cut, events on red filled area are accepted. Left area corresponds to events coming from the deuterium target. Right area corresponds to events coming from the solid target.}
\label{fig:vze}
\end{figure}
The target cuts are:
\begin{table}[!htb]
\centering
\begin{tabular}{|l|r|}
\hline
Target&Cut\\ \hline
C&-25.33<vz<-24.10\\ \hline
Fe&-25.65<vz<-24.26\\ \hline
Pb&-25.54<vz<-24.36\\ \hline
D2&-31.8<vz<-28.40\\ \hline
\end{tabular}
\label{tb:targCut}
\end{table}
The details on criteria of these cuts are in the appendix \ref{sec:TargetFits}.
At the end of this procedure we can classify the particle by target and kinematics.
\clearpage
\section{Charged pions and neutral pion channel}
\subsection{Charged pion identification}
For the charged pions, the analysis in \cite{Hayk_ana} is used as a reference. The toroidal field in CLAS makes the acceptance of positive and negative particles different: $\pi^+$ out-bend, $\pi^-$ in-bend.
The momentum is extracted from the EVNT bank, the charged pions energy sampling fraction on the EC is unknown and the momentum resolution is good enough to make possible the use of the energy momentum relation $E_{\pi^\pm} = \sqrt{|\vec{p_{\pi^\pm}}|^2 + m_{\pi^\pm}^2 }$, with $m_{\pi^\pm}=0.13957 GeV$.
A cut based on the arrival time to the SC of the $\pi$ candidates is applied:
\begin{equation}
\Delta t = \left( t_{sc}(\pi) -r_{sc}(\pi)/30*\sqrt{(m_\pi/p_{\pi})^2 +1}\right) - \left( t(e^-)_{sc} -r_{sc}(e^-)/30 + 0.08 \right)
\label{eq:pi_delta_t} 
\end{equation} 
Where $r_{sc}(\cdot)$ and $t_{sc}(\cdot)$ are the path and the time registered on SCPB respectively. The first parenthesis is the time difference of the $\pi$ candidate with respect to an ideal $\pi$, the second is the electron time taken as a reference, to make the distribution centered at 0, see Fig. \ref{fig:pi_delta_t}. This cut is applied depending on the pion momentum.
For $\pi^+$ and $\pi^-$ the cuts are in the tables \ref{tb:pip_DeltaCut} and \ref{tb:pim_DeltaCut} respectively. No cut on $\Delta t$ is used for $\pi^+$ with momentum bigger than $2.7 GeV$, instead, a minimum number of photo electrons in CC is required (CCPB.Nphe>25).
\begin{table}
\centering
\begin{tabular}{|l|l|}
\hline
Momentum range GeV & $\Delta t$ range ns\\ \hline
[0.0, 0.25] & [-1.45, 1.05]\\ \hline
[0.25, 0.75] & [-1.44, 1.05]\\ \hline
[0.75, 1.00] & [-1.40, 1.05]\\ \hline
[1.00, 1.25] & [-1.35, 1.03]\\ \hline
[1.25, 1.50] & [-1.35, 0.95]\\ \hline
[1.50, 1.75] & [-1.35, 0.87]\\ \hline
[1.75, 2.00] & [-1.25, 0.68]\\ \hline
[2.00, 2.25] & [-0.95, 0.65]\\ \hline
[2.25, 2.70] & [-1.05, 0.61]\\ \hline
\end{tabular}
\caption{$\Delta t$ cuts depending on the $\pi^+$ momentum range.}
\label{tb:pip_DeltaCut}
\end{table}
%
\begin{table}
\centering
\begin{tabular}{|l|l|}
\hline
Momentum range GeV & $\Delta t$ range ns\\ \hline
[0.00, 0.50] & [-0.87, 0.63]\\ \hline
[0.50, 1.00] & [-0.55, 0.37]\\ \hline
[1.00, 1.50] & [-0.55, 0.38]\\ \hline
[1.50, 2.00] & [-0.60, 0.44]\\ \hline
[2.00, 2.50] & [-1.00, 0.45]\\ \hline
[2.50, 3.00] & [-1.00, 0.40]\\ \hline
[3.00, $\infty$] & [-2.00, 0.45]\\ \hline
\end{tabular}
\caption{$\Delta t$ cuts depending on the $\pi^-$ momentum range.}
\label{tb:pim_DeltaCut}
\end{table}

\begin{figure}[!ht]
\centering
\includegraphics[scale=0.3]{pi_T4.gif}
\caption{Time difference distribution according to Eq. \eqref{eq:pi_delta_t}.}
\label{fig:pi_delta_t}
\end{figure}

The $\pi^+$ candidates with high momentum can be confused with protons, nevertheless the required minimum number of photo-electrons reject the protons. The selection of pions is shown  in Fig. \ref{fig:beta_pi}
%%
\begin{figure}[!ht]
\centering
\begin{subfigure}[b]{0.45\textwidth}
\includegraphics[width=\textwidth]{beta_P_pip.gif}
\caption{}
\label{fig:beta_pip}
\end{subfigure}
~
\begin{subfigure}[b]{0.45\textwidth}
\includegraphics[width=\textwidth]{beta_P_pim.gif}
\caption{}
\label{fig:beta_pim}
\end{subfigure}
\caption{$\beta$ over momentum for charged pions. The black dashed lines represent the predicted curve ($p/\sqrt(p^2 + m^2)$). \protect\subref{fig:beta_pip} positive pions.  \protect\subref{fig:beta_pim} negative pions.}
\label{fig:beta_pi}
\end{figure}
\clearpage
%%%%

%\begin{figure}[!ht]
%\centering
%\includegraphics[scale=0.3]{beta_p_pi_before.gif}
%\caption{Beta momentum dependency. The cut on $\Delta t$ is not included.}
%\label{fig:beta_p_pi_before}
%\end{figure}

The $\pi^-/e^-$ discrimination is performed by requiring a maximum number of photo-electrons in the CC (CCPB.Nphe<25). 

For the $\pi^-$ candidates, two cuts based on the deposit energy EC are implemented: A maximum limit in the total deposited energy registered in the ECPB bank (EVNT.Etot<0.15); $\pi^-$ must  lie in a region in the ECPB.Ein $\otimes$ ECPB.Etot plane, see Fig. \ref{fig:pim_ECPB}.
\begin{equation}
EVNT.Ein < 0.85 - 0.5EVNT.Eout
\label{eq:pim_ECPBCut}
\end{equation}
%
\begin{figure}[!ht]
\centering
\includegraphics[scale=0.3]{pim_EinEtotCut.gif}
\caption{The $\pi^-$ must lie below the black line.}
\label{fig:pim_ECPB}
\end{figure}

Having the pions identified, the inspection on the invariant mass spectrum shows just a small excess on the region corresponding to $\eta$ meson. The quantity $\Delta m = m(\pi^+,\pi^-,\pi^0) - m(\gamma,\gamma) - m(\pi^+) - m(\pi-) + 2*0.13957 + 0.135$, which is the total kinetic energy released in the center of mass, plus some mass parameters to shift the spectrum, was also estimated \cite{D_meson}.  See Fig. \ref{fig:m_pippimpi0}.
\begin{figure}[!ht]
\centering
\includegraphics[scale=0.7]{etaMassCnoK0rho.pdf}
\caption{Invariant mass  m($\pi^+,\pi^-,\gamma,\gamma$) with a cut $0.1<m(\gamma,\gamma)<0.2$.}
\label{fig:m_pippimpi0}
\end{figure}

The $\Delta m$ improves the signal to noise ratio but not enough to get a clean $\eta$ reconstruction, the level of background beneath the $\eta$ peak is too high.
There is not enough statistics to reconstruct the $\eta$ in different bins. Nevertheless the background over the different bins is reduced for higher $z$, see Fig. \ref{fig:Dm_evol}.

\begin{figure}[!ht]
\centering
\includegraphics[scale=0.25]{eta_3pi_Dm_evol.png}
\caption{$\Delta$ spectrum in different Z bins.}
\label{fig:Dm_evol}
\end{figure}

An estimation of the number of $\eta$ is done by counting events within the limits $0.4<\Delta m<0.6$
\clearpage

The $\eta$ meson yield was extracted from the invariant mass spectrum produced by the sum of four particles 4-momentums ($\pi^+,\pi^-,\gamma,\gamma$), the $\pi^0$ was implemented as a cut $0.1<m_{\gamma\gamma}<0.2$, the definition for the $\gamma$ momentum is according to Eq. \eqref{eq:P_ECPB}. The events selection must accomplish $N_{\pi^+}\geqslant 1 \cap N_{\pi^-}\geqslant 1 \cap N_{\gamma}\geqslant 2$, all possible combinations are $\eta$ candidates ($\binom{N_{\pi^+}}{1}\times\binom{N{\pi^-}}{1}\times\binom{N_{\gamma}}{2}$). For instance, having an event with $N_{\pi^+}=2$, $N_{\pi^-}=1$ and $N_{\gamma}=3$, the total of $\binom{2}{1}_{\pi^+}\times\binom{1}{1}_{\pi^-}\times\binom{3}{2}_{\gamma}=12$.

The invariant mass of the $\eta$ candidate is defined by:
\begin{equation}
M_{\eta}^2 = (p_{\pi^+}+p_{\pi^-}+p_{\pi^0}(\gamma\gamma))^2
\end{equation}
In the study of this channel the identification of $\pi^+$ and $\pi^-$ is required.
\clearpage
\section{Acceptance correction}
The detector acceptance is calculated using MC generated events, called from now on \textit{thrown events}, and the simulation reconstructed events, called from now on \textit{accepted events}. 
Let $N_s(\cdot)$,  $N_r(\cdot)$ and $N(\cdot)$ be the number of $\pi^0$: thrown by the particle generator, reconstructed from MC generated events and real data events; respectively. The number of $\eta$ corrected $N'(\cdot)$ is:
\begin{equation}
N'(\cdot) = \frac{N(\cdot)}{(N_r(\cdot)/N_s(\cdot))}
\end{equation}
Where $(\cdot)$ reflects the bin dependency (i.e. $(Q^2,\nu,z,p_T^2)$).

The particle generator used is Lepto64 \cite{lepto}, the chain of simulation is:\\
\textbf{lepto} $\rightarrow$ \textbf{gsim} $\rightarrow$ \textbf{gpp} $\rightarrow$ \textbf{user\_ana} $\rightarrow$ \textbf{DST}

Where \textbf{gsim} simulates the passage of the particle through the Clas detector; \textbf{gpp} corrects the tails and resolution of the distributions to make it similar to the ones observed in real data; \textbf{user\_ana} performs the reconstruction producing the \textbf{DST} files.

The decay of the $\eta$ was done inside \textbf{lepto} because of the short mean lifetime ($ct \sim 152pm$). The GSIM bank in the DST has only the information of the stable particles thrown ($\gamma$, $\pi^+$, $\pi^-$) not the mother particle, hence an invariant mass fit was required on the thrown events too.

Two sets of simulations where carried out:
\begin{description}
\item[\namedlabel{it:sima}{Sim A}] A simulation keeping all the generated particles from \textbf{lepto}. 
\item[\namedlabel{it:simb}{Sim B}] A simulation keeping only events which contains at least one eta in the produced particles. The selection was done at the \textbf{lepto} level.
\end{description}
%
The invariant mass of the two sets of simulations combined are shown in Fig. \ref{fig:M_sim}.
%%%
\begin{figure}[!ht]
\centering
\begin{subfigure}[b]{0.45\textwidth}
\includegraphics[width=\textwidth]{Meta_aa_sim.gif}
\caption{}
\label{fig:M_aa_sim}
\end{subfigure}
~
\begin{subfigure}[b]{0.45\textwidth}
\includegraphics[width=\textwidth]{Meta_3pi_sim.gif}
\caption{}
\label{fig:M_3pi_sim}
\end{subfigure}
\caption{The invariant mass of the two set of simulations (unfiltered and filtered $\eta$ events), the simulation/data is about 10 times. \protect\subref{fig:M_aa_sim} $\eta \rightarrow \gamma\gamma$. \protect\subref{fig:M_3pi_sim} $\eta \rightarrow \pi^{+} \pi^{-}\pi^{0}(\gamma\gamma)$ }
\label{fig:M_sim}
\end{figure}
\clearpage
\subsection{Two gamma decay channel}
\label{s:eta_aa_AC}
In this case the model used to estimate the number of $\eta$ is:
\begin{equation}
f(x) = N_{\eta}BW(x;M,\Gamma) \circledast G(x;\mu,\sigma) + N_b pol1(x;c_1)
\label{eq:pi0_gsim_model}
\end{equation}
Where $BW(x;M,\Gamma)$ is a Breit-Wigner distribution of a particle with mass $M$ and decay width $\Gamma$; $G(x;\mu,\sigma)$ is a normal distribution; and $pol1(x;c_1)$ is a normalized 1st-degree polynomial with slope $a_1$. The parameters $N_{\eta}$ and  $N_b$ save the number of events corresponding to the different parts of the model. See Fig. \ref{eq:pi0_gsim_model}.

The model prototype used in data (Eq. \eqref{eq:eta_aa_model}) is also used in accepted events, see Fig. \ref{fig:eta_aa_sim_fit}.

\begin{figure}[!ht]
\centering
\includegraphics[scale=0.3]{eta_aa_sim_fit.gif}
\caption{Accepted $\eta$ events. Upper plot: invariant mass spectrum of $2\gamma$, the fitted function correspond to eq. \eqref{eq:eta_aa_model}. Lower plot: plot of $(data(x)-model(x))/data_{error}(x)$ showing the fit performance.}
\label{fig:eta_aa_sim_fit}
\end{figure}
%
\begin{figure}[!ht]
\centering
\includegraphics[scale=0.7]{eta_aa_gsim_fit.png}
\caption{Thrown $\eta$ events. Upper plot: invariant mass spectrum of $2\gamma$, the red fitted function corresponds to eq. \eqref{eq:pi0_gsim_model}. Lower plot: plot of $(data(x)-model(x))/data_{error}(x)$ showing the fit performance.}
\label{fig:eta_aa_gsim_fit}
\end{figure}
\clearpage
%
\subsection{3 pion decay channel}
\label{s:eta_3pi_AC}
The number of events generated in the first set of simulations \ref{it:sima} was used to make background studies. The two set of simulations \ref{it:sima} and \ref{it:simb} were used in combination in the acceptance correction. \\

%
%\begin{figure}[!ht]
%\centering
%\includegraphics[scale=0.25]{eta_3pi_sim.png}
%\caption{Simulation reconstruction. Left: events thrown. Right: events reconstructed.}
%\label{fig:eta_3pi_sim}
%\end{figure}
%\clearpage
%
\subsection{Simulation and data comparison}
%%%
The distribution of the variables coming from the electrons are in good agreement ($Q^2$, $\nu$), the ones pertaining to the hadron are not so good ($p_T^2$, $z$), nevertheless all of them cover the phase space used in this analysis and hence make possible to use this simulation in the acceptance correction. All the variables spectra for $\eta \rightarrow \gamma\gamma$ are shown in Fig. \ref{fig:aa_simcmp} and the ones for $\eta \rightarrow \pi^+\pi^-\pi^0$ in Fig. \ref{fig:3pi_simcmp}.
\begin{figure}[!ht]
\centering
\begin{subfigure}[b]{0.45\textwidth}
\includegraphics[width=\textwidth]{Q2_eta_aa_simdata_cmp.gif}
\caption{}
\label{fig:Q2_aa_simcmp}
\end{subfigure}
~
\begin{subfigure}[b]{0.45\textwidth}
\includegraphics[width=\textwidth]{Nu_eta_aa_simdata_cmp.gif}
\caption{}
\label{fig:Nu_aa_simcmp}
\end{subfigure}
~
\begin{subfigure}[b]{0.45\textwidth}
\includegraphics[width=\textwidth]{Z_eta_aa_simdata_cmp.gif}
\caption{}
\label{fig:Z_aa_simcmp}
\end{subfigure}
~
\begin{subfigure}[b]{0.45\textwidth}
\includegraphics[width=\textwidth]{Pt2_eta_aa_simdata_cmp.gif}
\caption{}
\label{fig:Pt2_aa_simcmp}
\end{subfigure}
\caption{Comparison between reconstructed simulation (red) and data (black). All spectra are obtained after considering the limits \eqref{eq:abs_lim}. \protect\subref{fig:Q2_aa_simcmp} Q2 spectrum. \protect\subref{fig:Nu_aa_simcmp} $\nu$ spectrum. \protect\subref{fig:Z_aa_simcmp} z spectrum. \protect\subref{fig:Pt2_aa_simcmp} $p_T^2$ spectrum.}
\label{fig:aa_simcmp}
\end{figure}
\clearpage
%%%% 3pi simcmp
\begin{figure}[!ht]
\centering
\begin{subfigure}[b]{0.45\textwidth}
\includegraphics[width=\textwidth]{Q2_eta_3pi_simdata_cmp.gif}
\caption{}
\label{fig:Q2_3pi_simcmp}
\end{subfigure}
~
\begin{subfigure}[b]{0.45\textwidth}
\includegraphics[width=\textwidth]{Nu_eta_3pi_simdata_cmp.gif}
\caption{}
\label{fig:Nu_3pi_simcmp}
\end{subfigure}
~
\begin{subfigure}[b]{0.45\textwidth}
\includegraphics[width=\textwidth]{Z_eta_3pi_simdata_cmp.gif}
\caption{}
\label{fig:Z_3pi_simcmp}
\end{subfigure}
~
\begin{subfigure}[b]{0.45\textwidth}
\includegraphics[width=\textwidth]{Pt2_eta_3pi_simdata_cmp.gif}
\caption{}
\label{fig:Pt2_3pi_simcmp}
\end{subfigure}
\caption{Comparison between simulation reconstructed and data. All spectra are obtained after considering the limits \eqref{eq:abs_lim} and the mass cut in Fig. \ref{fig:M_eta_aa_stat}. \protect\subref{fig:Q2_3pi_simcmp} Q2 binning. \protect\subref{fig:Nu_3pi_simcmp} $\nu$ binning. \protect\subref{fig:Z_3pi_simcmp} z binning. \protect\subref{fig:Pt2_3pi_simcmp} $p_T^2$ binning.}
\label{fig:3pi_simcmp}
\end{figure}
\clearpage
%%%%%%%%%%%
%
\section{Coulomb correction for electrons}
Before and after the electron proton interaction take place, a long distance effect is produced by the charge of the nuclei. The Coulomb field affects the incoming and outgoing electrons, modifying its wave functions and hence its kinematics. The effect can be estimated at Born level using the Dirac equation and partial wave analysis and including the nuclei as a static potential perturbation. 
It has been proven that the enhancement of the electron momentum can be estimated using the so called "effective momentum approximation" (EMA) \cite{CC_2008}. In this approximation the change on the electron direction is neglected, only the magnitude of the momentum is changed. The effective potential is $V_0' = cV_0$ with $c \in [0.7,0.8]$\footnote{Here it was used $c=0.8$}, $V_0 = \frac{3\alpha(Z-1)}{2R}$ the electrostatic potential at nucleus center, $R$ is the nuclear radius \footnote{$R\approx 1.1A^{1/3} + 0.86A^{-1/3}$}, $Z$ is the atomic number and $\alpha$ the fine structure constant. The change is:
\begin{align}
k_{eff} &= k+\Delta k,~~~\Delta k = V_0'/c\\
k_{eff}'&= k'+\Delta k
\label{eq:CC}
\end{align}
See Fig.\ref{fig:CC_pic}.
\begin{figure}[!ht]
\centering
\includegraphics[scale=0.2]{CC_pic.png}
\caption{Coulomb correction for electrons.}
\label{fig:CC_pic}
\end{figure}
The corrections is estimated event by event and the effect on Pb data can be seen on Fig. \ref{fig:CC_pic_example}.
%
\begin{figure}[!ht]
\centering
\includegraphics[width=0.8\textwidth,keepaspectratio]{CC_example.png}
\caption{Coulomb correction effect observed in Lead data. The correction for the other targets is lower.}
\label{fig:CC_pic_example}
\end{figure}
\clearpage

\section{Radiative corrections} 
There are two types of radiative corrections that will be addressed in this analysis. The first one is related to the DIS electrons. The measured spectrum of the electron variables needs to be corrected due to QED processes that are not negligible, see Fig. \ref{fig:RC_FD}. In order to compensate the effect of the NLO Feynman diagrams, the peaking approximation is used \cite{Mo:1968cg}. In this technique, the photon emitted before and after are always parallel to the incident and scattered electron respectively. This allows the computation of the integral required to estimate the radiative cross section.
After the estimation of the radiative cross section, the ratio to the Born cross section is calculated \eqref{eq:delta_rc} and the yield of the electron variables are corrected with this number \eqref{eq:rc_corr}.
\begin{align}
\delta_{RC}(Q^2,\nu)&=\sigma_{rad}(Q^2,\nu)/\sigma_{born}(Q^2,\nu) \label{eq:delta_rc}\\
N_e(Q^2,\nu)_rc &= N_e(Q^2,\nu)/\delta_{RC}(Q^2,\nu)\label{eq:rc_corr}
\end{align}

The SIDIS events are also corrected. Here, the main difference is the inclusion of the final hadron in the structure functions of the interaction vertex, and consequently, a modification of the radiative cross section. The method used to calculate this cross section is based on \cite{Akushevich1999}, but instead of using fragmentation functions to describe the hadronization part of cross section, a fitting procedure on the $\phi_{pq}$ spectrum is used to estimate the structure functions directly from the data. The correction is carried out in the same way as for DIS, but correcting the hadron variables spectrum.

The Born cross section has the following dependence:
\begin{equation}
\sigma_0=\frac{d\sigma_0}{dxdydzdp_t^2d\phi_{pq}}=\frac{N}{Q^4}\lrp{A+A_{c}cos(\phi_{pq}) + A_{cc}cos(2\phi_{pq})},
\label{eq:born_xs}
\end{equation}

The coefficients $A$ do not depend on the $\phi_{pq}$ and they have the following form.

\begin{align}
A&=2Q^2\mathcal{H}_1 + \lrp{SX - M^2Q^2}\mathcal{H}_2 \nonumber \\
&+ \lrp{4a^1a^2 + 2b^2 - M_h^2Q^2}\mathcal{H}_3 \nonumber \\
&+ \lrp{2Xa^1+2Sa^2 - zS_xQ^2}\mathcal{H}_4 \nonumber\\
A_c &= 2b\lrp{2a^+\mathcal{H}_3 + S_p\mathcal{H}_4} \nonumber\\
A_{cc} &= 2b^2 \mathcal{H}_3 \label{eq:A}
\end{align}

The details of the other variables are in the equations in \cite{Akushevich1999}.

Solving the system of equations \eqref{eq:A} together with the measurement of $\sigma_L/\sigma_T \approx 0.14$ within the Hall-C of the TJNAF  \cite{rlt_hallC}, the structure functions (SF) can be estimated. The same SF are used in the radiative cross section, hence an iterative procedure can be used to estimate the radiative corrections, as follows: 

\begin{itemize}
\item[1] The $\phi_{pq}$ distribution is fitted with a model according to equation \eqref{eq:born_xs}. The coefficients $A$ are the result of the fit.
\item[2] The structure functions are calculated by solving the system of equations \eqref{eq:A} considering the value of $\sigma_L / \sigma_T = 0.14$
\item[3] The structure functions are used in the package HAPRAD \cite{Akushevich1999} and the radiative tail is estimated. Go to step 1 with the spectrum modified by the radiative corrections.  
\end{itemize}
%
\begin{figure}[!ht]
\centering
\includegraphics[width=0.8\textwidth,keepaspectratio]{Rad_Cor.png}
\caption{Schematic representation of the radiative processes.}
\label{fig:RC_FD}
\end{figure}
