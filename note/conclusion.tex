\chapter{Conclusions and future analysis}
\section{Analysis at 12GeV}
With the upgrade to CLAS12 the phase space,Fig. \ref{fig:CLAS12_update}, and the luminosity will allow a more refined analysis of the $\eta$ meson. An interesting contrast of two different channels can be carried out.
Having access to $x<0.1$ will allow to study the effect of the pre-hadron formation via $q\bar{q}$ or stuck quark.

Also, a Dalitz plot analysis can be carried out on the channels $\eta \rightarrow \pi^+\pi^-\pi^0$ and $\eta \rightarrow \pi^0\pi^0\pi^0$, this allows inspection on the QCD symmetry breaking, related with the difference of $m_u-m_d$ quark masses, \cite{Kloe_eta}. This was not possible with the available data, see Fig.  \ref{fig:dalitz_carbon}.
%
\begin{figure}[H]
\centering
\includegraphics[scale=0.25]{PS_12gev.png}
\caption{Phase space increased with the CLAS12 update.}
\label{fig:CLAS12_update}
\end{figure}
%
\begin{figure}[H]
\centering
\includegraphics[scale=0.25]{Dalitz_3pi_eta.gif}
\caption{Dalitz plot of $\eta \rightarrow \pi^+\pi^-\pi^0$, on carbon data.}
\label{fig:dalitz_carbon}
\end{figure}  