\chapter{Introduction}
\pagenumbering{arabic}
The hadron formation process is one of the unknown features of QCD. First, the process could be initiated by a struck quark or a $q\bar{q}$ diople. After the first interaction, the QCD system (pre-hadron) which will lead to the hadron involves low momentum transfer processes, hence the perturbative QCD techniques could not be applied to model the hadronization and non-perturbative QCD models are used, making use of parameters extracted from experiments.

On top of this, the QCD interactions does not allow to measure the quarks and gluons isolated, the cleanest interaction to study the hadrons is the electron-positron interaction ($e^-e^+ \rightarrow \gamma^*/Z \rightarrow q\bar{q} \rightarrow hX$), nevertheless the detector technology allows only the measurement of the final hadron. In order to have access to the hadronization process at a femtoscopic scale, DIS scattering with different nuclear environments is the best option. On the cleanest picture (electron-positron annihilation), fragmentation functions are defined to parametrize the evolution of partons into hadrons, the effects of the medium can be included through medium modified fragmentation functions, nevertheless this topic is still unsolved and different model exists \cite{FF_feynman, string_1,string_2,string_3_lund,precolor_1,precolor_2,precolor_3}, thus the measurement of different hadron production in DIS experiments are needed to improve the models and decide which one is more suitable to the reality.

\section{Measured Observable}
The great majority of nucleons are inside a nucleus of some atom. Hence, the understanding of the effect of the nuclear environment on the  hadronization process is a crucial matter.  

The multiplicity of an hadron $h$ produced in an electron-nucleon scattering is defined as the number of hadrons $h$ produced per DIS event. It can be estimated using the ratio of the cross section of the production of $h$ to the total DIS cross section:
\begin{equation}
M_h=\frac{1}{\sigma_{DIS}}\frac{d^2 \sigma (eN\rightarrow ehX)}{dQ^2dz}\overset{{\scriptscriptstyle Fact.}}{=}\frac{\int dx \sum_f e_f^2 q_f(x,Q^2) D_f^h (Q^2,z)}{ 
\int dx \sum_f e_f^2 q_f(x,Q^2)}
\label{eq:h_multi}
\end{equation}
Taking the multiplicity of a hadron in a given nucleus and comparing it with the multiplicity obtained for Deuterium (essentially no nuclear environment), the effects of the nuclear environment are exposed. From now on this ratio will be called Multiplicity Ratio.
\begin{equation}
R_h^A(Q^2,\nu,z,p_T^2)=\frac{\lrp{\frac{1}{\sigma}\frac{d^2 \sigma (eN\rightarrow ehX)}{dQ^2dz}}_A}{\lrp{\frac{1}{\sigma}\frac{d^2 \sigma (eN\rightarrow ehX)}{dQ^2dz}}_D}=\frac{\lrp{N_h(Q^2,\nu,z,p_T^2)/N_e(Q^2,\nu)}_A}{\lrp{N_h(Q^2,\nu,z,p_T^2)/N_e(Q^2,\nu)}_D} \label{eq:MR_raw}\\
\end{equation}
Where $N_h$ is the yield of the semi-inclusive hadron produced and $N_e$ is the number of DIS electrons, $\nu$ and $Q^2$ is the energy and virtuality of the  virtual photon.

The literature shows the behaviour for pions and kaons, some of the results for these hadrons are here \cite{Hermes_2007,Hermes_2011}. The kinematical region of the electron beam in the experimental data used in this analysis, Jlab E02-104, compared to the HERMES collaboration data (in this field) is shown in Fig. \ref{fig:hermes_cmp}.

\begin{figure}[!ht]
\centering
\includegraphics[scale=0.35]{kin_hermes_cmp.png}
\caption{Kinematical region used in the experiment E02-104 compared to Hermes data\cite{Hermes_2011}.}
\label{fig:hermes_cmp}
\end{figure}

There is no such measurement for $\eta$ particles, and neither any hadron has been measured in different decay channels.

This analysis is dedicated to the study of hadron formation at long distances (non-perturbative regime), and the effects of the nuclear environment on it, through the measurement of $\eta$ particles multiplicity ratio in deep inelastic scattering (DIS).

The $\eta$ meson has three relevant decay channels:

\begin{table}[ht!]
\centering
\begin{tabular}{|c|c|}
\hline
$\bm{\eta \rightarrow \gamma \gamma}$ &  39.41\%\\\hline
$\bm{\eta \rightarrow \pi^0 \pi^0 \pi^0}$ &  32.68\%\\\hline
$\bm{\eta \rightarrow \pi^0 \pi^+ \pi^-}$ &  22.92\%\\\hline
\end{tabular}
\label{tab:eta_decay}
\end{table}

Only the channels $\bm{\eta \rightarrow \gamma \gamma}$ and $\bm{\eta \rightarrow \pi^0 \pi^+ \pi^-}$ where considered in this analysis.

\nocite{LQCD_1,LQCD_2,LQCD_3}
%  
\section{Semi Inclusive Deep inelastic scattering}
In this analysis the hadron formation is studied through deep inelastic scattering, which consists of a lepton hitting a nucleon with sufficient energy to break it apart ($lN \rightarrow l'X)$. 

Let $k^\mu$ and $k'^\mu$ be four-momenta of the incoming and outgoing leptons, $P^\mu$ the four momentum of the nucleon, $q^\mu=k^\mu - k'^\mu$ the momentum transfer, then see Fig. \ref{fig:DIS} for the diagram of SIDIS. 

The standard variables used in the analysis are: 
\begin{align}
Q^2 &= -q^2 = -(k^\mu+p^\mu)^2 \overset{Lab}{=} 4E_bE'Sin^2(\theta/2)\\
\nu&=p^\mu q_\mu /M \overset{Lab}{=}E_b-E'\\
M^2 &= P^2\\
x_{Bj}&=\frac{Q^2}{2P^\mu q_\mu}\overset{Lab}{=}\frac{Q^2}{2M\nu}\\
y&=\frac{P^\mu q_\mu}{P^\mu k_\mu}\overset{Lab}{=}\frac{\nu}{E_b}\\
W^2&=(q^\mu + P^\mu)^2 \overset{Lab}{=} M^2- Q^2 + 2\nu M
\label{eq:DIS}
\end{align}
%\onslide<2->{p^\mu&=(E_h,\vec{p_h})\\}
%\onslide<3->{Z&=\frac{p^\mu P_\mu}{q^\mu P_\mu} \overset{Lab}{=} \frac{E_h}{\nu}\\}
%
\begin{figure}[!ht]
\centering
\begin{subfigure}[b]{0.45\textwidth}
\includegraphics[width=\textwidth]{DIS_schematic.png}
\caption{}
\label{fig:DIS}
\end{subfigure}
~
\begin{subfigure}[b]{0.45\textwidth}
\includegraphics[width=\textwidth]{SIDIS_schema.png}
\caption{}
\label{fig:SIDIS}
\end{subfigure}
\caption{\protect\subref{fig:DIS} Schematic representation of DIS scattering. \protect\subref{fig:SIDIS} Schematic representation of SIDIS events.}
\end{figure}
%
If the lepton is charged, then the process is mediated by a virtual photon, which will be the point of interest of this analysis, since the energies of the experimental setup used are well below $M_Z$.

Extra information can be inferred by measuring an extra particle in the final state; this process is called semi-inclusive DIS ($lN \rightarrow l'hX$), see Fig. \ref{fig:SIDIS}. Additional standard variables are used, $z,p_T^2,\phi_{pq}$:
%
\begin{align}
p^\mu&=(E_h,\vec{p_h})\\
z&=\frac{p^\mu P_\mu}{q^\mu P_\mu} \overset{Lab}{=} \frac{E_h}{\nu}\\
\left\|\vec{p_L}\right\|&= \frac{\vec{q}\cdot\vec{p_h}}{ \left\|\vec{q}\right\|}\notag \\
\vec{p_T}&=\vec{p_h} - \vec{p_L}\notag \\
p_T^2 &= (z\nu)^2 - m_h^2- \left( \frac{\left(\vec{q}\cdot\vec{p_h}\right)^2}{\nu^2 + Q^2}\right)
\label{eq:SIDIS}
\end{align}
The transverse momentum and angle are shown in Fig. \ref{fig:SIDIS_plane}
%
\begin{figure}[!ht]
\centering
\includegraphics[scale=0.3]{SIDIS_vars.png}
\caption{SIDIS  leptonic and hadronic planes definitions.}
\label{fig:SIDIS_plane}
\end{figure}
%
